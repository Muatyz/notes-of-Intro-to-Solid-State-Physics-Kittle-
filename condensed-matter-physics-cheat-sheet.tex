\documentclass[UTF8,a4paper,10pt,onecolumn]{ctexart}
\usepackage[left=0.1cm, right=0.2cm, top=0cm, bottom=0cm]{geometry}
%页边距
\CTEXsetup[format={\Large\bfseries}]{section} %设置章标题居左
\usepackage{amsmath, amsfonts, amssymb} % math equations, symbols
\usepackage[english]{babel}
\usepackage{color}	% color content
\usepackage{graphicx}	% import figures
\usepackage{url}	% hyperlinks
\usepackage{bm} 	% bold type for equations
\usepackage{multirow}
\usepackage{booktabs}
\usepackage{epstopdf}
\usepackage{epsfig}
\usepackage{algorithm}
\usepackage{algorithmic}
\usepackage{listings}
\usepackage{xcolor}
\usepackage{booktabs}
\usepackage{zhnumber}
\usepackage{longtable}
\usepackage{subfigure}
\usepackage{float}
\usepackage{caption}
\usepackage{subfigure}
\renewcommand\thesection{\zhnum{section}}
\renewcommand\thesubsection{\arabic{section}}
\renewcommand{\algorithmicrequire}{ \textbf{Input:}}
% use Input in the format of Algorithm  
\renewcommand{\algorithmicensure}{ \textbf{Initialize:}}
% use Initialize in the format of Algorithm  
\renewcommand{\algorithmicreturn}{ \textbf{Output:}}
% use Output in the format of Algorithm
%%简化编写公式难度
\usepackage{braket} %量子算符宏包
\usepackage{cancel}%表示项的消去 
\usepackage{siunitx} %单位宏包
\usepackage{microtype}%调整字间距,字间距的调整会影响到公式的间距
\usepackage{listings}
\usepackage{color}
\usepackage{fancyhdr} %设置页眉、页脚
\lhead{}
\chead{}
\lfoot{}
\cfoot{}
\rfoot{}
\usepackage{setspace}
\setstretch{0.3}
\usepackage{hyperref} %bookmarks
\hypersetup{colorlinks, bookmarks, unicode} %unicode

\begin{document}
\textbf{晶格}\textbf{(2)}sc(简单,$2r=a$);bcc($4r=\sqrt{3}a$,$\rho=\frac{2m_0}{a^3},a=\sqrt[3]{\frac{2m_0}{\rho}}$);fcc($4r=\sqrt{2}a$);hcp(六角密)\textbf{(3)常见结构}:NaCl(Cl面心\&角+Na边中\&体心);CsCl(Cs体心+Cl角);金刚石结构(fcc+$000\&\frac{1}{4}\frac{1}{4}\frac{1}{4}$);ZnS结构(金刚石替换)(Zn$000,0\frac{1}{2}\frac{1}{2},\frac{1}{2}0\frac{1}{2},\frac{1}{2}\frac{1}{2}0$;S$\frac{1}{4}\frac{1}{4}\frac{1}{4},\frac{1}{4}\frac{3}{4}\frac{3}{4},\frac{3}{4}\frac{1}{4}\frac{3}{4},\frac{3}{4}\frac{3}{4}\frac{1}{4}$)[例].$r_{\text{Cs}}=1.7,r_{\text{Cl}}=1.81.a=2(r_{\text{Cs}}+r_{\text{Cl}})/\sqrt{3},PF=\frac{4\pi(r_{\text{Cs}}^3+r_{\text{Cl}}^3)}{3a^3}\approx 0.682$,$\rho=\frac{m_{\text{Cs}}+m_{\text{Cl}}}{a^3}=4.2$;[例].NaCl下CsCl:$a=2(r_{\text{Cs}}+r_{\text{Cl}}),PF=\frac{4\pi(r_{\text{Cs}}^3+r_{\text{Cl}}^3)\cdot 4}{3a^3}\approx 0.525$;\textbf{两种指标}晶面截距$a_1,a_2,a_3$(1)$(a_1^{-1}a_{2}^{-1}a_2^{-1})$;(2)$[a_1,a_2,a_3]$.上划线为负$[u\overline{v}w]$.\textbf{布拉格条件}$2d\sin{\theta}=n\lambda;\Delta\vec{k}=\vec{G};2\vec{k}\cdot\vec{G}=\vec{G}^2$;\textbf{劳厄条件}$\forall i,\vec{a_i}\cdot\Delta\vec{k}=2\pi v_i$;\textbf{倒格子基}
  $\vec{b_1}=2\pi\frac{\vec{a_2}\times\vec{a_3}}{\vec{a_1}\cdot\vec{a_2}\times\vec{a_3}}$,
  $\vec{b_2}=2\pi\frac{\vec{a_3}\times\vec{a_1}}{\vec{a_1}\cdot\vec{a_2}\times\vec{a_3}}$,
  $\vec{b_3}=2\pi\frac{\vec{a_1}\times\vec{a_2}}{\vec{a_1}\cdot\vec{a_2}\times\vec{a_3}}$.
  $\vec{b_i}\cdot\vec{a_j}=2\pi\delta_{ij}$;\textbf{倒格矢}
  $\vec{G}=v_1\vec{b_1}+v_2\vec{b_2}+v_3\vec{b_3},v_{i}\in\mathcal{Z}$.倒格矢$\vec{G}_{h_1h_2h_3}\perp$$h_1h_2h_3$(实空间晶面).面间距$d=\frac{2\pi }{|\vec{G}_{\vec{h}}|}$\textbf{几何结构因子}前提:方向为$\vec{k'}=\vec{k}+\Delta\vec{k}=\vec{k}+\vec{G}$,
  $S_{G}=\sum_{j} f_{j}e^{-i\vec{r_{j}}\cdot\vec{G}}=\sum_{j}f(j)e^{-i2\pi(x_j v_1+y_j v_2+z_j v_3)}$,$f_{j}=\int dVn_{j}(\vec{r})e^{-i\vec{G}\cdot\vec{r}}$.[例].bcc$\&$$(0,0,0)+(\frac{1}{2},\frac{1}{2},\frac{1}{2}),S(v_1,v_2,v_3)=f(1+e^{-i\pi(v_1+v_2+v_3)})$;[例].fcc$\&$$(0,0,0)+(0,\frac{1}{2},\frac{1}{2})+(\frac{1}{2},0,\frac{1}{2})+(\frac{1}{2},\frac{1}{2},0)$,$S(v_1,v_2,v_3)=f\{1+e^{-i\pi(v_2+v_3)}+e^{-i\pi(v_1+v_3)}+e^{-i\pi(v_1+v_2)}\}$;\textbf{原子形状因子}$f_j=\int dVn_j(\vec{r})e^{-i\vec{G}\cdot\vec{r}}$,球对称极限$f_j=4\pi\int drn_j(r)r^2\frac{\sin{Gr}}{Gr}$;\textbf{第一布里渊区}倒格子的维格纳-塞茨原胞.晶格常数$a$(1)sc$\rightarrow$sc($2\pi/a$);bcc$\rightarrow$棱形12面体(长对角线$2\cdot\frac{\sqrt{2}\pi}{a}$,短对角线$2\cdot\frac{\pi}{a}$);fcc$\rightarrow$截角8面体(8面体每个角被切,使相邻3面的正方形边围成正6边形.小正方形和6边形的边长$l=\frac{\sqrt{2}\pi}{2a}$)\textbf{声子-振动}\textbf{(1)无阻尼单原子链:}$u_{s\pm 1}=ue^{isKa}exp^{\pm iKa}$,色散$w^2=(\frac{2C}{M})(1-\cos{Ka})=\omega_m^2\sin^2{\frac{1}{2}Ka};w^2=(\frac{4C}{M})\sin^2{\frac{1}{2}Ka}$;态密度:$D(\omega)=\frac{Na}{\pi}/|\partial_K\omega|,\partial_K\omega=\frac{a}{2}\omega_m\cos{\frac{1}{2}Ka}=\frac{a}{2}(\omega_m^2-\omega^2)^{\frac{1}{2}}$群速:$v_{g}=\partial_{K}\omega=\sqrt{\frac{Ca^{2}}{M}}\cos{\frac{Ka}{2}}$;[长波($Ka\ll 1):w^2=(\frac{C}{M})K^{2}a^{2},v=\frac{w}{K}$离散转连续:$M\partial_t^2u_s=\sum_pC_p(u_{s+p}-u_s)=\sum_{p>0}C_{p}[(u_{s+p}-u_s)+(u_{s-p}-u_s)]=\sum_{p>0}C_p\{[u(x+pa,t)-u(x,t)]+[u(x-pa,t)-u(x,t)]\}=\sum_{p>0}C_pp^2a^2\partial_x^2u(x,t)$,试探解$u_{s+p}=ue^{-i[\omega t-(s+p)Ka]}$,$\omega^2=\frac{2}{M}\sum_{p>0}C_p(1-cos{pKa})\approx K^2(a^2/M)\sum_{p>0}p^2C_p=v^2K^2\rightarrow\partial_t^2u=v^2\partial_x^2u$]\textbf{(2)无阻尼双原子链}:原胞p个原子,3个声支,3p-3个光支.$M_1\frac{d^2 u_s}{dt^2}=C(v_s+v_{s-1}-2u_s);M_2\frac{d^2 v_s}{dt^2}=C(u_{s+1}+u_{s}-2v_s)$.试解$u_s=ue^{isKa}e^{-iwt},v_s=ve^{isKa}e^{-iwt}$,系数行列式0:$M_1 M_2 w^4-2C(M_1+M_2)w^2+2C^2(1-\cos{Ka})=0$;长波($Ka\ll 1$):光支$w^2=2C(\frac{1}{M_1}+\frac{1}{M_2})$,声支$w^2=\frac{C}{2(M_1+M_2)}K^2 a^2$;光支下原子反向震动(质心固定),光电场激发.\textbf{(3)波矢选择定则}:波矢$\vec{k}$非弹性散射到$\vec{k'}$,同时产生/吸收$\vec{K}$的声子:$\vec{k}=\vec{k'}\pm\vec{K}+\vec{G},\vec{G}$是倒格矢;\textbf{(4)声子能量:}$\epsilon=(n+\frac{1}{2})\hbar\omega$.若$u=u_0\cos{Kx}\cos{\omega t},E_k=\int\frac{1}{2}\rho(\frac{\partial u}{\partial t})^2=\frac{1}{4}\rho V\omega^2u_0^2\langle\sin^2{\omega t}\rangle=\frac{1}{8}\rho V\omega^2u_0^2=\frac{1}{2}(n+\frac{1}{2})\hbar\omega$;动能守恒:$\frac{\hbar^2 k^2}{2M_n}=\frac{\hbar^2 k'^2}{2M_n}\pm\hbar\omega$\textbf{(5)有阻尼单原子链}:$m\partial_t^2u_j=C(u_{j+1}+u_{j-1}-2u_j)-\Gamma\partial_tu_j$.色散关系:$\omega(k)=\sqrt{\omega_{k_0}^2-(\frac{\Gamma}{2m})^2}-\frac{i\Gamma}{2m}(\omega_{k_0}=\sqrt{\frac{4C}{m}}|\sin{\frac{ka}{2}}|)$弛豫时间(a)$\omega_{k_0}\geq\Gamma/2m:\tau_k=2m/\Gamma$;(b)$\omega_{k_D}<\Gamma/2m:\tau_k=\frac{\Gamma}{2m\omega_{k_0}^2}(1+\sqrt{1-(\frac{2m\omega_{k_0}}{\Gamma})^2})$\textbf{(6)2D正方}:$M\partial_{t}^2{u_{l,m}}=C[(u_{l+1,m}+u_{l-1,m}-2u_{l,m})+(u_{l,m+1}+u_{l,m-1}-2u_{l,m})]$.设$u_{l,m}=u_0e^{i(lK_xa+mK_ya-\omega t)}$,色散关系$\omega^2 M=2C(2-\cos{K_{x}a}-\cos{K_{y}a})$(a)$K=K(1,0),\omega^2=\frac{2C}{M}(1-\cos{Ka})$;(b)$K=K(1,1)/\sqrt{2},\omega^2=\frac{4C}{M}(1-\cos{\frac{1}{\sqrt{2}}Ka})$,长波($Ka\ll 1$)$\omega^2\approx\frac{Ca^2}{M}(K_x^2+K_y^2)$,群速$v=\frac{\partial\omega}{\partial K}=\sqrt{\frac{Ca^2}{M}}$.\textbf{(7)变C等M双原子链}.$M\partial_t^2u_s=C(v_{s-1}-u_s)+10C(v_s-u_s),M\partial_t^2v_s=10C(u_s-v_s)+C(u_{s+1}-v_s)$.试$u_s=ue^{isKa}e^{-i\omega t},v_s=ve^{isKa}e^{-i\omega t}.|^{M\omega^2-11C,C(10+e^{-iKa})}_{C(e^{iKa+10}),M\omega^2-11C}|=0$,$\omega_{\pm}^2=\frac{C}{M}[11\pm\sqrt{121-20(1-\cos{Ka})}]$\textbf{(8)}$\omega=\omega(K),K=\omega^{-1}(\omega)$,轨道总数$N(\omega)=(\frac{L}{2\pi})^3\frac{4\pi}{3}K^3$,态密度$D(\omega)=|\partial_\omega N|$\textbf{热学基础}
  \textbf{(0)}$\frac{\Delta a}{a}=\frac{\Delta V}{3V},C_V=(\frac{\partial U}{\partial T})_V$,声子温度$\tau=k_{B}T$,晶格内能$U_{lat}\sum_{K}\sum_{p}\langle n_{K,p}\rangle\hbar\omega_{K,p}$\textbf{(1)普朗克分布}$\langle n\rangle=(e^{\hbar\omega/\tau}-1)^{-1}$\textbf{(2)}$U=\sum_{K}\sum_{p}\hbar\omega_{K,p}(e^{\hbar\omega_{K,p}/\tau}-1)^{-1}=\sum_{p}\int d\omega D_{p}(\omega)\hbar\omega (e^{\hbar\omega/\tau}-1)^{-1}$,
  $C_{lat}=k_{B}\sum_{p}\int d\omega D_{p}(\omega)\frac{x^2 e^x}{(e^x -1)^2}$($x=\frac{\hbar\omega}{\tau}=\hbar\omega/k_{B}T$),$D(\omega)$即为态密度\textbf{(3)}一维$D(\omega)$:$L=Na$,每个间隔$\Delta K=\frac{\pi}{L}$内一个模式,每个$K$三个偏振态(两横一纵)$D(\omega)d\omega=\frac{L}{\pi}\frac{dK}{d\omega}d\omega=\frac{L}{\pi}\frac{d\omega}{d\omega/dK}$(色散关系$\omega(K)$)\textbf{(4)}三维$D(\omega)$:$\forall i, K_{i}=\pm\frac{2n\pi}{L}$,$\vec{K}$单位体积内模式数$(\frac{L}{2\pi})^3=\frac{V}{8\pi^3}$,每种偏振模式总数$N=(\frac{L}{2\pi})^3(\frac{4\pi K^3}{3})$,态密度$D(\omega)=\frac{dN}{d\omega}=(\frac{VK^2}{2\pi^2})(\frac{dK}{d\omega})$
  \textbf{德拜模型}
  \textbf{(0)石墨烯模型(2D)}.C-C距$d$,声速$v$,晶格常数$a=\sqrt{3}d$,原胞面积$A=\frac{\sqrt{3}a^2}{2}$,德拜波矢$\pi k_D^2=\frac{(2\pi)^2}{A}$,德拜频率$\omega_D=vk_D$,德拜温度$\theta_D=\frac{\hbar\omega_D}{k_B}=\frac{\hbar v k_D}{k_B}$.$(\theta_D|_{d=1.42\r{A}}=2.13\times10^3K)$(1)3D,每种偏振声速恒定,$\omega=vK$,态密度$D(\omega)=\frac{V\omega^2}{2\pi^{2}v^3}$,德拜/截止频率$\omega_{D}^3=6\pi^{2}v^{3}N/V$,截止波矢$K_{D}=\omega_{D}/v=(6\pi^{2}\frac{N}{V})^{\frac{1}{3}}$,单偏振态内能$U_i=\int d\omega D(\omega)\langle n(\omega)\rangle\hbar\omega=\int_{0}^{\omega_{D}}d\omega(\frac{V\omega^{2}}{2\pi^{2}v^{3}})(\frac{\hbar\omega}{e^{\frac{\hbar\omega}{\tau}}-1})$,总内能$U=3U_i=\frac{3V\hbar}{2\pi^2 v^3}\int_{0}^{\omega_{D}}d\omega\frac{\omega^3}{e^{\frac{\hbar\omega}{\tau}}-1}=\frac{3Vk_{B}^{4}T^{4}}{2\pi^2 v^3 \hbar^3}\int_{0}^{x_{D}}dx\frac{x^3}{e^x - 1}(x=\frac{\hbar\omega}{\tau},x_{D}=\frac{\hbar\omega_{D}}{\tau}=\frac{\theta}{T}$),德拜温度$\theta=\frac{\hbar v}{k_{B}}(\frac{6\pi^2N}{V})^{\frac{1}{3}}$,$U=9Nk_{B}T(\frac{T}{\theta})^3\int_{0}^{x_D}dx\frac{x^3}{e^x -1}$[例]金刚石(3D)C-C距离$d$,声速$v$,晶格常数$a=\frac{4d}{\sqrt{3}}$,原胞体积$\Omega=\frac{a^3}{4}$,德拜波矢$\frac{4}{3}\pi k_D^3=\frac{(2\pi)^3}{\Omega}$,德拜温度$\theta_D=\frac{\hbar\omega_D}{k_B}=\frac{\hbar v k_D}{k_B}.(\theta_D|_{d=1.54\r{A}}=2.39\times10^3K)$(2)低温($T^3)(\int_{0}^{\infty}dx\frac{x^3}{e^x -1}=\frac{\pi^4}{15}$):$U\approxeq3\pi^2 Nk_{B}T^4/5\theta^3$,热容$C_{V}\approxeq\frac{12\pi^4}{5}Nk_{B}(\frac{T}{\theta})^3\approxeq 234Nk_{B}(\frac{T}{\theta})^3$;\textbf{爱因斯坦}$N\omega_0$振子($D(\omega)=N\delta(\omega-\omega_{0})$):1D:$U=N\langle n\rangle\hbar\omega=N\frac{\hbar\omega}{e^{\hbar\omega/\tau}-1}$,1D:$C_{V}=(\frac{\partial U}{\partial T})_{V}=Nk_{B}(\frac{\hbar\omega}{\tau})^2\frac{e^{\hbar\omega/\tau}}{(e^{\hbar\omega/\tau}-1)^2}$.3D再乘3.
   \textbf{声子热学}\textbf{(1)态密度}一般:$D(\omega)=\frac{V}{(2\pi)^3}\int_{\text{K中}\partial\omega=0}\frac{dS_{\omega}}{v_{g}}$\textbf{(2)非谐作用}($U(x)=cx^2-gx^3-fx^4$):平均位移$\langle x\rangle=\frac{\int_{-\infty}^{+\infty}dx xe^{-\beta U(x)}}{\int_{-\infty}^{+\infty}dx e^{-\beta U(x)}}$($\beta=\frac{1}{k_{B}T}$),$\int dx xe^{-\beta U}\approxeq(\frac{3\pi^{\frac{1}{2}}}{4})(\frac{g}{c^{\frac{5}{2}}})\beta^{-\frac{3}{2}},\int dxe^{-\beta U}\approxeq(\frac{\pi}{\beta c})^{\frac{1}{2}}$,$\langle x\rangle=\frac{3g}{4c^2}k_{B}T$\textbf{(3)热导}.1D下热流量$j_{U}=-K\frac{dT}{dx}$,热导率$K=\frac{1}{3}Cvl$($C$:单位体积比热;$v$:粒子平均速度;$l$:平均自由程).\textbf{(4)过程}.$\vec{K}_{1}+\vec{K}_{2}=\vec{K}_{3}+\vec{G}$.正常($N$):$\vec{G}=0$;倒逆($U$):$\vec{G}\neq 0$
   \textbf{自由电子}
  \textbf{(0)一维无限深井}:$\mathcal{H}\psi_{n}=-\frac{\hbar^2}{2m}\frac{d^2\psi_{n}}{dx^2}=\epsilon_{n}\psi_{n};\epsilon_{n}=\frac{\hbar^2}{2m}(\frac{n\pi}{L})^2$\textbf{(1)费米能}$\epsilon_{F}$:N电子系统基态下最高能级;[例]一维无限深井+泡利原理:$2n_F=N,n=n_F,\epsilon_F=\frac{\hbar^2}{2m}(\frac{N\pi}{2L})^2$;\textbf{(2)温度变量}.$f(\epsilon,T,\mu)=(e^{[\epsilon-\mu(T)]/k_{B}T}+1)^{-1}$($T=0$时$\mu=\epsilon_F$).高温极限:玻尔兹曼分布/麦氏分布.\textbf{(3)(a)3D:}$-\frac{\hbar^2}{2m}\nabla^2\psi_{k}(\vec{r})=\epsilon_{\vec{k}}\psi_{k}(\vec{r}),\psi_{\vec{k}}(\vec{r})=e^{i\vec{k}\cdot\vec{r}},(\forall i,k_i=\frac{2n\pi}{L}),\epsilon_{\vec{k}}=\frac{\hbar^2}{2m}(k_x^2+k_y^2+k_z^2)$.
  $\hat{p}\psi_{\vec{k}}(\vec{r})=\hbar\vec{k}\psi_{\vec{k}}(\vec{r}),\vec{v}=\frac{\hbar\vec{k}}{m}$.F波矢$k_F$,F能$\epsilon_F=\frac{\hbar^2k_{F}^2}{2m}.K$空间体积元$(\frac{2\pi}{L})^3$存在单波矢$(k_x,k_y,k_z)$.F球+泡利定理:$2\cdot\frac{4\pi k_F^2/3}{(2\pi/L)^3}=N$.F波矢$k_F=(\frac{3\pi^2N}{V})^{\frac{1}{3}}=(3\pi^2 n)^{\frac{1}{3}}$,F能$\epsilon_{F}=\frac{\hbar^2}{2m}(\frac{3\pi^2N}{V})^{\frac{2}{3}}=\frac{\hbar^2}{2m}(3\pi^2n)^{\frac{2}{3}}$,F速度$v_F=(\frac{\hbar k_F}{m})=\frac{\hbar}{m}(\frac{3\pi^2N}{V})^{\frac{1}{3}}$.F温度$T_F=\epsilon_F/k_B$.态密度$N(U\leq\epsilon)=\frac{V}{3\pi^2}(\frac{2m\epsilon}{\hbar^2})^{\frac{3}{2}},D(\epsilon)=\frac{dN}{d\epsilon}=\frac{V}{2\pi^2}(\frac{2m}{\hbar^2})^{\frac{3}{2}}\epsilon^{\frac{1}{2}}=\frac{3N}{2\epsilon}$,\textbf{0K}:$U_0=2\sum_{k<k_F}\frac{\hbar^2k^2}{2m}$,\textbf{K}中状态数体密度为$\frac{V}{8\pi^3}$,$\frac{U_0}{V}=\frac{2}{8\pi^3}\int_{k<k_F}d^3k\frac{\hbar^2k^2}{2m}=\frac{1}{\pi^2}\frac{\hbar^2k_F^5}{10m},N=2\cdot\frac{4\pi k_F^3}{3}\frac{V}{8\pi^3}$,$U_0=\frac{3}{5}N\epsilon_F$,压强$P=-(\partial_VU_0)_N=-\frac{3}{5}(\partial_V\epsilon_F)_N=\frac{2}{3}\frac{U_0}{V}$,体模量$B=-V(\partial_VP)=-V\partial_V[\frac{N\hbar^2}{5m}(\frac{3\pi^2N}{V})^{\frac{2}{3}}\frac{1}{V}]=\frac{10}{9}\frac{U_0}{V}$\textbf{(b)2D:}$2\pi k_F^2\frac{A}{(2\pi)^2}=N,k_F=\sqrt{\frac{2\pi N}{A}}=\sqrt{2\pi n}$.色散:$\epsilon=\frac{\hbar^2 k^2}{2m},d\epsilon=\frac{\hbar^2kdk}{m}$.态密度$D(\epsilon)d\epsilon=2\frac{2\pi kdk}{A}\cdot\frac{A}{(2\pi)^2}=\frac{kdkd\epsilon}{\pi d\epsilon}=\frac{md\epsilon}{\pi\hbar^2},n=\int_{-\infty}^{+\infty}D(\epsilon)n_F(\epsilon)d\epsilon=\frac{m}{\pi\hbar^2}\int_{0}^{+\infty}\frac{d\epsilon}{e^{(\epsilon-\mu)/k_BT}+1}$$=\frac{mk_BT}{\pi\hbar^2}\ln{(e^{\mu/k_BT}+1)}$.化学势$\mu(T)=k_BT\ln(e^{\frac{\pi n \hbar^2}{mk_BT}}-1)$\textbf{(4)比热容}.总e内能$U_{e}\approx\frac{NT}{T_F}k_B T$,e比热$C_{e}=\frac{\partial U}{\partial T}\approx Nk_B \frac{T}{T_F}$.低温($k_B T\ll \epsilon_F$):$\Delta U=\int_0^{\infty}d\epsilon\epsilon D(\epsilon)f(\epsilon)-\int_0^{\epsilon_F}d\epsilon\epsilon D(\epsilon)$$=\int_{\epsilon_F}^{\infty}d\epsilon(\epsilon-\epsilon_F)f(\epsilon)D\epsilon + \int_0^{\epsilon_F}d\epsilon(\epsilon_F-\epsilon)[1-f(\epsilon)]D(\epsilon)$.e热容$C_e=\frac{dU}{dT}=\int_{0}^{\infty}d\epsilon(\epsilon-\epsilon_F)\frac{df}{dT}D(\epsilon)\approx D(\epsilon_F)\int_0^{\infty}d\epsilon(\epsilon-\epsilon_F)\frac{df}{dT}$低温($\tau=k_B T,x=\frac{\epsilon-\epsilon_F}{\tau})\int_{-\infty}^{+\infty}dxx^2\frac{e^x}{(e^x +1)^2}=\frac{\pi^2}{3},C_e=\frac{1}{3}\pi^2D(\epsilon_F)k_B^2T$($D(\epsilon_F)=\frac{3N}{2\epsilon_F}$),$C_e=\frac{\pi^2Nk_BT}{2T_F}$.\textbf{(5)金属比热}.$\frac{C}{T}=\gamma+AT^2(\gamma$索末菲常量).\textbf{(6)电导率}.$\vec{F}=-e(\vec{E}+\frac{1}{c}\vec{v}\times\vec{B})$.若$\vec{F}=-e\vec{E},\delta\vec{k}=-e\vec{E}t/\hbar,\vec{v}=\delta\vec{k}/m=-e\vec{E}\tau/m$.电流密度$\vec{j}=nq\vec{v}=ne^2\tau\vec{E}/m$,($\vec{j}=\sigma\vec{E}$)$\sigma=\frac{ne^2\tau}{m},\rho=\sigma^{-1}$.[电子漂移速度v:$m(\partial_tv+v/\tau)=-eE$,试$E=E_0e^{-i\omega t},v=v_0e^{-i\omega t}$,$v=\frac{-(1+i\omega\tau)}{1+(\omega\tau)^2}\frac{e\tau}{m}E,\sigma(\omega)=j/E=-env/E=\frac{e^2\tau n}{m}(\frac{1+i\omega\tau}{1+(\omega\tau)^2})$]\textbf{(7)磁场下运动}.(CGS制)$\hbar(\frac{d}{dt}+\frac{1}{\tau})\delta\vec{k}=\vec{F}=-e(\vec{E}+\vec{v}\times\vec{E})$.若$\vec{B}=B\hat{z}$,$\{v_x=-\frac{e\tau}{m}E_x-\omega_c\tau v_y,v_y=-\frac{e\tau}{m}E_y+\omega_c\tau v_x,v_z=-\frac{e\tau}{m}E_z\}$,回旋频率$\omega_c=\frac{eB}{mc}$[漂移速度理论:$m(\partial_t+\tau^{-1})v_x=-e(E_x+\frac{B}{c}v_y),m(\partial_t+\tau^{-1})v_y=-e(E_y-\frac{B}{c}v_x),m(\partial_t+\tau^{-1})v_z=-eE_z$,$j=-nev$,$v_x=\frac{1}{1+(\omega_c\tau)^2}(-\frac{e\tau}{m}E_x+\frac{\omega_c\tau^2e}{m}E_y),v_y=\frac{1}{1+(\omega_c\tau)^2}(-\frac{\omega_c\tau^2e}{m}E_x-\frac{e\tau}{m}E_y),v_z=-\frac{e\tau}{m}E_z$,$[j_x,j_y,j_z]^{T}=\frac{\sigma_0}{1+(\omega_c\tau)^2}[1,-\omega_c\tau,0;\omega_c\tau,1,0;0,0,1+(\omega_c\tau)^2][E_x,E_y,E_z]^{T},\sigma_0=\frac{ne^2\tau}{m},\omega_c=\frac{Be}{mc}$.若$j_y=0,E_y=-\omega_c\tau E_x,j_x=\sigma_0E_x$:自由e理论too simple]
  \textbf{(8)霍尔效应}.霍尔系数$R_H=\frac{E_y}{j_x B}=-\frac{1}{nec}$(CGS)\textbf{(9)金属热导率}.$K_e=\frac{1}{3}Cvl=\frac{\pi^2}{3}\frac{nk_B^2T}{mv_F^2}v_Fl=\frac{\pi^2 n k_B^2 T\tau}{3m}$(10)洛伦兹常量$L=\frac{K}{\sigma T}=\frac{\pi^2}{3}(\frac{k_B}{e})^2=2.45\times 10^{-8}(W\Omega/\text{deg}^2)$.\textbf{(10)金属受力}自由电子.($n_{Cu}\approx 10^{6}$)$k_F=(3\pi^2n)^{\frac{1}{3}}\propto n^{\frac{1}{3}};\epsilon_F=\hbar^2 k_F^2/2m=\hbar^2(3\pi^2n)^{\frac{2}{3}}\propto n^\frac{2}{3},D(\epsilon)\propto\epsilon^{\frac{1}{2}}.\langle\epsilon\rangle=\frac{\int_0^{\epsilon_F}\epsilon D(\epsilon)d\epsilon}{\int_0^{\epsilon_F}D(\epsilon)d\epsilon}=\frac{3}{5}\epsilon_F,E=N\langle\epsilon\rangle\propto V^{-\frac{2}{3}},P=-\frac{dE}{dV}=\frac{2}{5}n\epsilon_F\propto\epsilon_F^{\frac{5}{2}}.\frac{dP}{d\epsilon_F}=n\rightarrow\Delta P\approx n\Delta \epsilon_F$.\textbf{(11)求第一布里渊区能带}3D$\epsilon(\vec{K})=\frac{\hbar^2}{2m}[(K_x+g_1\frac{2\pi}{a})^2+(K_y+g_2\frac{2\pi}{a})^2+(K_z+g_3\frac{2\pi}{a})^2],$\textbf{近自由电子模型}\textbf{(0)1D布拉格衍射条件}$(\vec{k}+\vec{G})^2=\vec{k}^2\rightarrow k=\pm\frac{1}{2}G=\pm\frac{n\pi}{a}$(倒格矢$G=\frac{2\pi n}{a}$)\textbf{(1)驻波}$!t:\psi(+)=e^{\frac{i\pi x}{a}}+e^{-\frac{i\pi x}{a}}=2\cos{\frac{\pi x}{a}},\psi(-)=e^{\frac{i\pi x}{a}}-e^{-\frac{i\pi x}{a}}=2i\sin{\frac{\pi x}{a}}.\rho(+)=|\psi(+)|^2\propto\cos^2{\frac{\pi x}{a}},\rho(-)=|\psi(-)|^2\propto\sin^2{\frac{\pi x}{a}}$.大小关系:$\langle\psi(-)|U|\psi(-)\rangle\leq\langle e^{\mp i\pi x/a}|U|e^{\pm i\pi x/a}\rangle\leq\langle\psi(+)|U|\psi(+)\rangle$.若1D:$\psi(x)=\sqrt{2}\cos{\frac{\pi x}{a}},\sqrt{2}\sin{\frac{\pi x}{a}}$,e势能$U(x)=U\cos{\frac{2\pi x}{a}}$,则1级近似能隙$E_g=U(+)-U(-)=\int_{0}^{1}dx U(x)[|\psi(+)|^2-|\psi(-)|^2]=U$.\textbf{(2)布洛赫函数}.若周期势,则$\psi_{\vec{k}}(\vec{r})=u_{\vec{k}}(\vec{r})e^{i\vec{k}\cdot\vec{r}}$(其中$u_{\vec{k}}(\vec{r})=u_{\vec{k}}(\vec{r}+\vec{T})$).若非简并,$\psi(x+a)=C\psi(x),C=e^{i2\pi s/N}\rightarrow\psi(x)=u_{\vec{k}}(x)e^{i2\pi sx/N}$.\textbf{(3)KP模型Kölnig-Penney}(周期$\delta$势阱).$-\frac{\hbar^2}{2m}\frac{d^2\psi}{dx^2}+U(x)\psi=\epsilon\psi$.
  $x\in(0,a):\psi=Ae^{iKx}+B^{-iKx},\epsilon=\frac{\hbar^2 K^2}{2m};x\in(-b,0):\psi=Ce^{Qx}+De^{-Qx},U_0-\epsilon=\frac{\hbar^2 Q^2}{2m}$.$\psi$连续+$\psi'$连续,有四阶系数行列式为0:$[(Q^2-K^2)/2QK]\sinh{Qb}\sin{Ka}+\cosh{Qb}\cos{Ka}=\cos{k(a+b)}$.取极限$b=0,U_0=\infty(Q\gg K,Qb\ll 1)$,即为周期性$\delta$函数,$P=\frac{Q^2ba}{2}$结论化为$(P/Ka)\sin{Ka}+\cos{Ka}=\cos{ka}$.带宽:$\frac{P}{\pi+\theta}(-\theta+\frac{\theta^{3}}{6})+(-1+\frac{\theta^{2}}{2})=\cos{ka}(\theta=-\frac{\pi}{P}(1+\cos{ka})),\Delta E=\frac{2\pi^{2}\hbar^{4}}{m^{2}a^{4}V_{0}}$.
  \textbf{(4)周期势下的电子波函数}.$U(x)=\sum_{G}U_{G}e^{iGx}$,若为实则$U(x)=\sum_{G>0}2U_G\cos{Gx}$.$\psi=\sum_{k}C(k)e^{ikx}$.
  波动方程$\sum_{k}\frac{\hbar^2}{2m}k^2C(k)e^{ikx}+\sum_{G}\sum_{k}U_G C(k)e^{i(k+G)x}=\epsilon\sum_{k}e^{ikx}$.
  中心方程$(\lambda_k-\epsilon)C(k)+\sum_{G}U_GC(k-G)=0$(其中$\lambda_k=\frac{\hbar^2 k^2}{2m}$)
  \textbf{(5)}$\det\{\{\lambda_{k-g}-\epsilon,U,0\},\{U,\lambda_{k}-\epsilon,U\},\{0,U,\lambda_{k+g}-\epsilon\}\}$.每一个$k$每个$\epsilon$在不同能带.\textbf{(6)中心方程求解K-P}(周期$\delta$势函数).$U(x)=Aa\sum_{s}\delta(x-sa),U_G=\int_0^{1}dxU(x)cos(Gx)=A.$中心方程变为$(\lambda_k-\epsilon)C(k)+Af(k)=0$,其中$f(k)=\sum_{n}C(k-2\pi n/a)=f(k\pm 2\pi n/a)$.从而有$\frac{mAa^2}{2\hbar^2}(Ka)^{-1}\sin{Ka}+\cos{Ka}=\cos{ka}$.极限$P\ll 1,$
  \textbf{(7)BZ近边界近似解}.$k^2=(\frac{1}{2}G)^2,(k-G)^2=(\frac{1}{2}G-G)^2, k=\pm\frac{1}{2}G$.($k=\frac{1}{2}G,\lambda=\hbar^2(\frac{1}{2}G)^2/2m$)$(\lambda-\epsilon)C(\pm\frac{1}{2}G)+UC(\mp\frac{1}{2}G)=0$.行列式$|^{\lambda-\epsilon,U}_{U,\lambda-\epsilon}|=0$,得$\epsilon=\lambda\pm U,E_g=2U$.若在$\frac{1}{2}G$附近,则$(\lambda_k-\epsilon)C(k)+UC(k-G)=0,(\lambda_{k-G})C(k-G)+UC(k)=0(\lambda_k=\frac{\hbar^2k^2}{2m}),|^{\lambda_k-\epsilon,U}_{U,\lambda_{k-G}-\epsilon}|=0,\epsilon=\frac{1}{2}(\lambda_{k-G}+\lambda_k)\pm[\frac{1}{4}(\lambda_{k-G}-\lambda_k)^2+U^2]^\frac{1}{2}$.以$\widetilde{K}=k-\frac{1}{2}G$展开,有$\epsilon_{\widetilde{K}}\approx\frac{\hbar^2}{2m}(\frac{1}{4}G^2+\widetilde{K}^2)\pm U[1+2(\frac{\lambda}{U^2})(\frac{\hbar^2\widetilde{K}^2}{2m})]$.
  \textbf{(8)轨道数.N原胞一维}:$k=\pm\frac{2n\pi}{L}$.每原胞一个k+泡利定理,每能带2N轨道.
  \textbf{(9)正方晶格}$U(x)=-4U\cos{\frac{2\pi x}{a}}\cos{\frac{2\pi y}{a}},\vec{r}=x\hat{i}+y\hat{j},\vec{G}=G_1\hat{b_1}+G_2\hat{b_2}=\frac{2\pi}{a}(g_1\hat{b_1}+g_2\hat{b_2}).$;$U(\vec{r})=-U(e^{i\frac{2\pi}{a}x}+e^{-i\frac{2\pi}{a}x})(e^{i\frac{2\pi}{a}y}+e^{-i\frac{2\pi}{a}y})=-U[e^{i\frac{2\pi}{a}(x+y)}+e^{i\frac{2\pi}{a}(x-y)}+e^{-i\frac{2\pi}{a}(x-y)}+e^{-i\frac{2\pi}{a}(x+y)}]=U_{G(11)}e^{iG(11)\cdot \vec{r}}+U_{G(\overline{1}1)}e^{iG(\overline{1}1)\cdot \vec{r}}+U_{G(1\overline{1})}e^{iG(1\overline{1})\cdot \vec{r}}+U_{G(\overline{11})}e^{iG(\overline{11})\cdot\vec{r}}=\sum_{G(11)}e^{iG(11)\cdot\vec{r}}$.中心方程$(\lambda_k-\epsilon)C(\vec{K})+U_G(11)C(\vec{K}-\vec{G}(11))+U_{\vec{G}}(\overline{11})C(\vec{K}-\vec{G}(\overline{1}1))+U_G(1\overline{1})C(\vec{K}-\vec{G}(1\overline{1}))+U_G(\overline{1}1)C(\vec{K}-\vec{G}(\overline{1}1))$.若$\vec{K}=\vec{G}(\frac{1}{2}\frac{1}{2})=\frac{1}{2}\vec{G}(11),|^{\lambda_{\frac{1}{2}G(11)}-\epsilon,-U}_{-U,\lambda_{-\frac{1}{2}G(11)}-\epsilon}|=0,
  \epsilon=\frac{\hbar^2\pi^2}{ma^2}\pm U$
   \textbf{紧束缚}\textbf{(1)}$E(\vec{k})=\epsilon_{i}-\sum_{s}J(\vec{R}_s)e^{-i\vec{k}\cdot\vec{R}_s}(\vec{R}_s=\vec{R}_n-\vec{R}_m)$\textbf{(2)1D,s}:$E(\vec{k})=\epsilon_{s}-J_{0}-J_{1}e^{-ika}-J_{1}e^{ika}=\epsilon_{s}-J_{0}- 2J\cos{ka}$;\textbf{(3)2D,sc:}$E = \epsilon - 2t(\cos(k_{x}a) + \cos(k_{y}a))$;
  Honeycomb:$\phi(\vec{r}) = c_{A}\phi_{A}(\vec{r}) + c_{B}\phi_{B}(\vec{r})= \frac{1}{\sqrt{N}}\sum_{\vec{R}_{m}}e^{i\vec{k}\cdot\vec{R}_{m}}[c_{A}\varphi(\vec{r}-\vec{R}_{m}^{A}) + c_{B}\varphi(\vec{r}-\vec{R}_{m}^{B})],E(\vec{k}) = \epsilon_{1}\pm J\sqrt{3+2\cos{(\sqrt{3}k_{y}a)}+4\cos{(\frac{\sqrt{3}k_{y}a}{2})}\cos{(\frac{3k_{x}a}{2})}}$(4)3D,(sc):$\epsilon(\vec{k}) = \epsilon_{s} - J_{0} - 2J_{1}(\cos{k_{x}a}+\cos{k_{y}a}+\cos{k_{z}a})$;(bcc):$\epsilon(\vec{k}) = -\alpha - 8\gamma\cos{(\frac{k_{x}a}{2})}\cos{(\frac{k_{y}a}{2})}\cos{(\frac{k_{z}a}{2})}$;(fcc)$\epsilon(\vec{k}) = -\alpha - 4\gamma[\cos{(\frac{k_{y}a}{2})}\cos{(\frac{k_{z}a}{2})}+\cos{(\frac{k_{z}a}{2})}\cos{(\frac{k_{x}a}{2})}+\cos{(\frac{k_{x}a}{2})}\cos{(\frac{k_{y}a}{2})}]$
\textbf{(5)简并:}$\phi(\vec{r}) = \frac{1}{\sqrt{N}}\sum_{\vec{R}_{m}}\sum_{j}e^{i\vec{k}\cdot\vec{R}_{m}}c_{j}\varphi_{j}(\vec{r}-\vec{R}_{m})$
\textbf{(6)周期δ势}.$V(x)=-aV_{0}\delta(x),-\frac{\hbar^{2}}{2m}\psi''=E\psi,\psi=\sqrt{k}e^{-k|x|}(k=\sqrt{\frac{-2mE}{\hbar^{2}}}),-\frac{\hbar^{2}}{2m}\nabla^{2}\psi'(0)-aV_{0}\psi(0)=0,k=\frac{mV_{0}a}{\hbar^{2}},E=-\frac{mV_{0}^{2}a^{2}}{2\hbar^{2}},\psi=\sqrt{\frac{mV_{0}a}{\hbar^{2}}}e^{-\frac{mV_{0}a}{\hbar^{2}}|x|}$.TB条件:$k\ll a$.通解$E(k)=\epsilon-J(0)-\sum_{nn}J(R_{m})e^{ikR_{m}},J(R_{m})=\sum_{R_{m}\neq 0}aV_{0}\int\rho^{*}\delta(r-R_{m})\rho(r-a)\mathrm{d}r=\frac{mV_{0}^{2}a^{2}}{\hbar^{2}}\sum_{n=1}^{\infty}e^{-\frac{mV_{0}a}{\hbar^{2}}(2na-a)}\approx \frac{mV_{0}^{2}a^{2}}{\hbar^{2}}e^{-\frac{mV_{0}a^{2}}{\hbar^{2}}}.E(k)=\epsilon--2J(R_{m})\cos{kR_{m}}-J(0)$.能隙$\Delta E=4J=\frac{4mV_{0}^{2}a^{2}}{\hbar^{2}}e^{-\frac{mV_{0}a^{2}}{\hbar^{2}}}$
   \textbf{近自由电子}\textbf{(1)非简并}$\varphi_{k}(x) = \varphi_{k}^{0}(x) + \sum_{k'(k'\neq k)}\frac{\langle k'|V(x)|k\rangle}{E_{k}^{0} - E_{k'}^{0}}\varphi_{k'}(x)(\langle k'|V(x)|k\rangle = \frac{1}{L}\int e^{-i(k'-k)x}V(x)\mathrm{d}x=V_{G}(G=k'-k)),\varphi_{k} = \varphi_{k}^{0}(x) + \sum_{k'(k'\neq k)}\frac{\langle k'|V(x)|k\rangle}{E_{k}^{0}-E_{k'}^{0}}\varphi_{k'}^{0}(x)= \varphi_{k}^{0}(x) + \sum_{k'(k'\neq k)}\frac{V_{G}}{E_{k}^{0} - E_{k'}^{0}}\varphi_{k'}^{0}(x)$,1级:$\langle k|V(x)|k\rangle = \frac{1}{L}\int_{0}^{L}V(x)\mathrm{d}x = \overline{V} = 0$;2级:$E_{k}^{2} = \sum_{k'}\frac{|\langle k'|V(x)|k\rangle|^{2}}{E_{k}^{0} - E_{k'}^{0}} = \sum_{G}\frac{|V_{G}|^{2}}{\frac{\hbar^{2}}{2m}\left[k^{2}-\left(k + G\right)^{2}\right]}$.(I)if $(k + G)^{2} \gg k^{2}$,自由e;(II)$(k + G)^{2} = k^{2}$\textbf{(2)简并}$\{^{(E_{k}^{0}-E)a+V_{G}^{*}b=0}_{V_{G}a+(E_{k'}^{0}-E)b=0}$, $|^{(E_{k}^{0}-E),V_{G}^{*}}_{V_{G},(E_{k'}^{0}-E)}|=0$,$E_{k\pm} = \frac{1}{2}\{(E_{k}^{0} + E_{k'}^{0})\pm\sqrt{(E_{k}^{0} + E_{k'}^{0})^{2} + 4|V_{G}|^{2}}\}$.(I)$E_{k}^{0} = E_{k'}^{0}\text{(BZ边界):}E_{k\pm} = E_{k}^{0}\pm|V_{G}|$;(II)$|E_{k}^{0} - E_{k'}^{0}|\gg|V_{G}|\text{(远离BZ):}E_{k\pm}=\{^{E_{k'}^{0} + \frac{|V_{G}|^{2}}{E_{k'}^{0} - E_{k}^{0}}}_{E_{k}^{0}  + \frac{|V_{G}|^{2}}{E_{k'}^{0} - E_{k}^{0}}}\text{(III)靠近BZ边界}(|E_{k}^{0} - E_{k'}^{0}|\ll|V_{G}|$):$E_{k\pm} = \frac{1}{2}\left\{E_{k}^{0} + E_{k'}^{0} \pm\left[2|V_{G}| + \frac{(E_{k'}^{0} - E_{k}^{0})^{2}}{4|V_{G}|}\right]\right\}$,$E_{k}^{0} + E_{k'}^{0}=2E_{0} + \frac{\hbar^{2}}{m}\left(k + \frac{G}{2}\right)^{2},(E_{k'}^{0} - E_{k}^{0})^{2}=4\left(\frac{\hbar^{2}}{2m}\right)^{2}G^{2}\left(k + \frac{G}{2}\right)^{2}$,即$E_{k\pm} \approx (E_{0}\pm |V_{G}|) + \frac{\hbar^{2}}{2m}\left(k + \frac{G}{2}\right)^{2} \pm \left(\frac{\hbar^{2}}{2m}\right)^{2} \frac{G^{2}}{2|V_{G}|}\left(k + \frac{G}{2}\right)^{2}$.\textbf{(3)2D简单正方:}$V(\vec{r})=-\sum_{n}a^{2}V_{0}\delta(\vec{r}-\vec{R}_{n})$.近似条件:势能很小,视为微扰($|U_{G}|=V_{0}\ll \frac{\hbar^{2}}{2m}(\frac{\pi^{2}}{a^{2}})$),能隙$2|V_{G=G_2}|=|\frac{1}{a^{1}}\iint Ve^{i\vec{G}_{2}\cdot\vec{r}}\mathrm{d}\vec{r}|=2V_{0}$
   \textbf{半导体}价带顶($\bigcap$),导带底($\bigcup$)\textbf{(1)e群速度}:$\vec{v}_{g} = \nabla_{\vec{k}}\omega(\vec{k})= \frac{1}{\hbar}\nabla_{\vec{k}}E(\vec{k})\textbf{(2)有效质量.}\frac{1}{m^{*}} = \frac{1}{\hbar^{2}}\frac{\mathrm{d}^2 E}{\mathrm{d}k^2}$,方向:$(\frac{1}{m{*}})_{\mu\nu}=\frac{1}{\hbar^{2}}\frac{\mathrm{d}^{2}E}{\mathrm{d}k_{\mu}\mathrm{d}k_{\nu}}$;另一种定义:$m^{*} = \hbar^{2}k\left(\frac{\partial E}{\partial k}\right)^{-1}$,线性色散$E = a(|\vec{k}-\vec{k}_{0}|)$:$m^{*} = \frac{\hbar|\vec{k}|}{v_{g}} = \frac{\hbar}{v_{g}}|\vec{k}-\vec{k}_{0}|$.能隙$\Delta = 2m_{0}v_{g}^{2}$.\textbf{(3)空穴}:$\vec{k}_{h} = -\vec{k}_{e};E_{h}(\vec{k}_{h}) = - E_{e}(\vec{k}_{e});\vec{v}_{h} = -\frac{1}{\hbar}\nabla_{\vec{k}_{h}}E_{h}(\vec{k}_{h}) = \frac{1}{\hbar}\nabla_{\vec{k}_{e}}E_{e}(\vec{k}_{e}) = \vec{v}_{e}$.\textbf{(4)激子}$\frac{1}{\mu^{*}} = \frac{1}{m_{C}^{*}} + \frac{1}{m_{hh}^{*}},(m_{C}^{*} = 0.067m_{e},  m_{hh}^{*} = 0.45m_{e})$,长度$a_{0}^{*} = \frac{\epsilon_{r}m_{e}}{\mu^{*}}\cdot a_{0}(a_{0} = \frac{\epsilon_{0}h^{2}}{\pi m_{e}e^{2}} \approx 0.53\text{Å})$\textbf{(5)粒子浓度}:$\mathrm{d}n = f(E, T)g(E)\mathrm{d}E(f(E,T) = \frac{1}{1+e^{(E-\mu)/k_{B}T}})$.$E - \mu\gg k_{B}T$极限F-D 分布退化为 B 分布:$f(E, T) \approx e^{-(E-\mu)/k_{B}T}$.导带找到e:$f_{C}\approx e^{-\frac{(E-\mu)}{k_{B}T}}$;价带找到h:$f_{h} = 1 - f_{V}=e^{-(\mu-E)/k_{B}T}.g(E)$因近似抛物线色散($E - E_{C} = \frac{(k-k_{C})^{2}}{2m_{C}^{*}},E - E_{V} =-\frac{(k-k_{V})^{2}}{2m_{h}^{*}}$),即态密度:$g_{C}(E) = a(m_{C}^{*})^{\frac{3}{2}}(E - E_{C})^{\frac{1}{2}};g_{V}(E) = a(m_{h}^{*})^{\frac{3}{2}}(E_{V} - E)^{\frac{1}{2}}$,$n = \int_{E_{C}}^{\infty}f_{C}g_{C}\mathrm{d}E\approx a(m_{C}^{*})^{\frac{3}{2}}\int_{E_{C}}^{\infty}(E-E_{C})^{\frac{1}{2}}e^{-\frac{E-\mu}{k_{B}T}}\mathrm{d}E= N_{c} e^{-\frac{E_{C}-\mu}{k_{B}T}}(N_{C} = 2(\frac{k_{B}}{2\pi\hbar^{2}})^{\frac{3}{2}}(m_{C}^{*}T)^{\frac{3}{2}})$,同理 $p \approx N_{V}e^{-\frac{\mu - E_{V}}{k_{B}T}}(N_{V} = 2(\frac{k_{B}}{2\pi\hbar^{2}})^{\frac{3}{2}}(m_{h}^{*}T)^{\frac{3}{2}})$.\textbf{Law of Mass Action:}$np\approx WT^{3}e^{-\frac{E_{g}}{k_{B}T}}$(前提:$|\mu-E|\gg k_{B}T$)\textbf{(6)化学势}本征半导体$n$=$p$, $\frac{N_{V}}{N_{C}} = e^{\frac{2\mu-E_{C}-E_{V}}{k_{B}T}};\mu = \frac{1}{2}(E_{C}+ E_{V}) + \frac{3}{4}k_{B}T\ln{\frac{m_{h}^{*}}{m_{C}^{*}}}$\textbf{(7)电导率}(I)载流子迁移率($\mu_{e}, \mu_{h}$)$\mu = \frac{|v|}{E}$(电荷 $q$ 漂移速度 $v =\frac{q\tau E}{m},\tau$为碰撞时间)$\mu_{e/h} = \frac{e\tau_{e/h}}{m_{e/h}}$.半导体:$\sigma = ne\mu_e + pe\mu_h$\textbf{(7)掺杂}半导体原子价态为$\nu$.$n$型掺杂(杂质为$\nu+1$);p型掺杂(杂质为 $\nu-1$)(I)浅掺杂能级:类H.(i)掺杂e:$E_{d}=-\frac{m_{c}^{*}}{m_{e}}\frac{1}{\epsilon_{r}^{2}\times\frac{13.6\text{eV}}{n^{2}}}$;(2)掺杂h:$E_{a}=-\frac{m_{V}^{*}}{m_{e}}\frac{1}{\epsilon_{r}^{2}}\times\frac{13.6\text{eV}}{n^{2}}$.参与导电.\textbf{(8)非本征载流子浓度}掺杂较少,$\mu$还在$E_{g}$中:$np=WT^{3}e^{-\frac{E_{g}}{k_{B}T}}$.全电离:$N-p=N_{D}-N_{A}$.\text{半导体pn结}.p型:$\mu(E_{F})$比$E_{i}$更近价带顶;n型:$\mu(E_{F})$比$E_{i}$ 更近导带底.e,h扩散,通过$\mu(E_{F})$拉平.(I)金-半导体:(i)肖特基:半导体$\mu(E_{F})\uparrow$,e从半导体到金属;内建$\vec{E}$,导带$E\uparrow$,导带底和费米能级距离$\uparrow$(ii)欧姆:半导体$\mu(E_{F})$低于金属,对e无势垒.(II)金-氧化绝缘体-半(MOS)$E_{F}$独立,类电容.(i)正电压:e从半导体远端到绝缘端.$\mu(E_{F})$更近导带底, n型强化;(ii)反电压:e向半导体远端移动.超限后,发生反型($\mu(E_{F})$更近价带顶)
   \textbf{(9)布洛赫振荡}运动方程$\hbar\frac{\mathrm{d}k}{\mathrm{d}t}=-eE$,解$k(t)=k(0)-\frac{eE}{\hbar}t$.色散$\epsilon(k)=\epsilon_{0}[1-\cos{(ak)}]$,e群速度$v(k)=\frac{1}{\hbar}\frac{\mathrm{d}\epsilon}{\mathrm{d}k}=\frac{\epsilon_{0}a}{\sin{(ak)}}$.$k(0)=x(0)=0$,$x(t)=\int_{0}^{t}v[k(t')]\mathrm{d}t'=\frac{\epsilon_{0}}{eE}[\cos{(\frac{eEa}{\hbar}t)}-1]$.$\omega_{\text{BO}}=eEa/\hbar$.观测条件$\tau\gg 2\pi/\omega_{\text{BO}}=h/eEa$.
   \textbf{布洛赫电子动力学}\textbf{(1)运动特征}$\frac{\hbar\mathrm{d}\vec{k}}{\mathrm{d}t}=-e\vec{v}\times\vec{B}$.e群速度$\vec{v}=\frac{1}{\hbar}\nabla_{k}E(\vec{k});\frac{\mathrm{d}E}{\mathrm{d}t}=\nabla_{k}(\vec{k})\frac{\mathrm{d}\vec{k}}{\mathrm{d}t}=0$.实空间和倒空间运动方向垂直.\textbf{(2)回旋频率}$(k_{z}=0)$.周期$T=\frac{2\pi K}{\frac{evB}{\hbar}}=\frac{2\pi}{eB}\frac{\hbar K}{v}=\frac{2\pi m}{eB}$;回旋频率$\omega_{c} = \frac{2\pi}{T} = \frac{eB}{m_{c}^{*}}(m_{c}^{*}\ncong m^{*})$\textbf{(3)磁场中分立能}(1.抛物线色散2.忽略自旋)朗道能级$E(k) = \frac{\hbar^{2}}{2m}k_{z}^{2} + \left(n + \frac{1}{2}\right)\hbar\omega_{c}$.(I)简并度(i)无磁场:$E(\vec{k}) = \frac{\hbar^{2}}{2m}(k_{x}^{2} + k_{y}^{2})$(ii)有磁场:相邻朗道环$L_{n},L_{n+1}$所围态简并.态数目$n_{k} = \Delta A \times \frac{S}{4\pi^{2}}=\pi\left[\Delta(k_{x}^{2} + k_{y}^{2})\right]\times \frac{S}{4\pi^{2}}= \frac{2\pi m\Delta E}{\hbar^{2}}\times \frac{S}{4\pi^{2}}=\frac{2\pi m\hbar\omega_{c}}{\hbar^{2}}\times \frac{S}{4\pi^{2}}= \frac{4\pi^{2}eB}{h}\times \frac{S}{4\pi^{2}} = \frac{eBS}{\hbar}$,朗道能级简并度$p = 2n_{k} = \frac{2e}{h}BS = \frac{BS}{\Phi_{0}}(\Phi_{0} = \frac{h}{2e} \approx 2.067\times 10^{-15} (\text{Wb}))$.高量子态条件:$\oint\vec{p}\cdot\mathrm{d}\vec{r} = (n + \gamma)\cdot 2\pi\hbar,A_{r} = \frac{2\pi\hbar}{eB}(n + \gamma)$.$\hat{B} \times \frac{\mathrm{d}\vec{k}}{\mathrm{d}t} = -\frac{eB}{\hbar}\frac{\mathrm{d}\vec{r}_{\perp}}{\mathrm{d}t},\frac{A_{k}}{A_{r}} = (\frac{eB}{\hbar})^{2}, A_{k} = \frac{2\pi eB}{\hbar}(n + \gamma)$.或$\frac{1}{B} = \frac{2\pi e}{\hbar A_{k}}(n + \gamma),\Delta(\frac{1}{B}) = (\frac{1}{B_{n+1}} - \frac{1}{B_{n}})=\frac{2\pi e}{\hbar}\frac{1}{A_{k}}$\textbf{(II)de Hass-Van Alphen}(i)2D:$\left|_{\blacktriangleright}^{\vartriangleright}\right|$横轴为磁矩,横轴为磁场.$\Delta(\frac{1}{B}) = \frac{2\pi e}{\hbar} \frac{1}{A_{k_{F}}},A_{F}$极值轨道;(ii)fcc(Au,Ag,Cu)$n = \frac{4}{a^{3}},k_{F} = (3\pi^{2}n)^{\frac{1}{3}} = (\frac{12\pi^{2}}{a^{3}})^{\frac{1}{3}}\approxeq 4.90 a^{-1},\text{跨越BZ最短距离:}\sqrt{3} b = (\frac{2\pi}{a})\approxeq 10.88a^{-1},\frac{4\pi}{a}\approxeq 12.57a^{-1},\text{Au:}\Delta(\frac{1}{B}) = 2\times 10^{-9} \text{G}^{-1},\text{极值轨道:}S = \frac{2\pi e}{\hbar}[\Delta(\frac{1}{B})]^{-1}\approxeq 4.8\times 10^{16}\text{cm}^{-2}$\textbf{(4)磁场下2D电子}$E = (n + \frac{1}{2})\hbar\omega_{c}$(I)展宽(i)本征:$\delta E \approx \frac{\hbar}{\tau}$,分辨条件$\omega_{c}\tau \gg 1$;(ii)$T$:分辨条件$\hbar\omega_{c} > k_{B}T$(低温)(II)简并度:单位面积内每个朗道能级的e数.单位面积内朗道能级简并度:$n_{L} = \frac{2eB}{h}$(i)$\sigma$极小(态密度谷):$N_{L} = n\frac{2eB}{h}$,(ii)$\sigma$极大(态密度峰):$N_{L} = (n + \frac{1}{2})\frac{2eB}{h}$,$\sigma$周期:$\Delta(\frac{1}{B}) = \frac{2e}{hN_{L}}$(III)霍尔效应.$\overleftrightarrow{\sigma} = \frac{\sigma_{0}}{1+(\omega_{c}\tau)^{2}}\left[^{1,-\omega_{c}\tau}_{\omega_{c}\tau,1}\right]$,系数$R_{H} = \frac{E_{y}}{j_{x}B} = \frac{\sigma_{xy}}{\sigma_{xx}^{2} + \sigma_{xy}^{2}}\frac{1}{B}=\frac{-1}{B}\frac{\omega_{c}\tau}{\sigma_{0}} = -\frac{1}{ne}$.2D各向同性:$\left[_{J_{y}}^{J_{x}}\right]=\left[^{\sigma_{xx},\sigma_{xy}}_{-\sigma_{xy},\sigma_{yy}}\right]\left[^{E_{x}}_{E_{y}}\right]$,霍尔效应:$\left\{^{\rho_{xx} = \frac{E_{x}}{J_{x}} = \frac{\sigma_{xx}}{\sigma_{xx}^{2} + \sigma_{xy}^{2}}}_{\rho_{xy} = \frac{E_{y}}{J_{x}} = \frac{\sigma_{xy}}{\sigma_{xx}^{2} + \sigma_{xy}^{2}}}\right\}$.极限$\omega_{c}\tau\gg 1,\sigma_{xy} \gg \sigma_{xx}:\left\{^{\rho_{xx} = \frac{\sigma_{xx}}{\sigma_{xy}^{2}}}_{\rho_{xy} = \frac{1}{\sigma_{xy}} = R_{H}B}\right\};R_{H} = \frac{E_{y}}{J_{x}B}$.\textbf{(5)电阻}(I)e-声子散射(准弹性散射)$E_{k'} = E_{k} \pm \hbar\omega,\vec{k}' =\vec{k} \pm\vec{q} + \vec{G}$.弛豫和散射概率:$\frac{1}{\tau} = (\frac{1}{2\pi})^{3}\int\omega_{\vec{k}, \vec{k}'}(1-\cos{\theta})\mathrm{d}\vec{k}'$(i)高T($T > \theta_{D}):\rho\propto\frac{1}{\tau}\propto T(\text{高温})$(ii)低T($q\ll k_{F}):N_{\text{声子}} \propto T^{3}(\text{低温}):1-\cos{\theta} = 2\sin^{2}\left(\frac{\theta}{2}\right) = \frac{1}{2}(\frac{q}{k_{F}})^{2}(\theta\text{很小})$.低温条件:$q\approx E \approx k_{B}T,1-\cos{\theta}\approx T^{2},\omega_{\vec{k},\vec{k}'}\propto T^{3}:\rho\propto \frac{1}{\tau}\propto T^{5}$(II)剩余电阻率(杂质).$\left(\frac{\partial \rho}{\partial T}\right)_{T\rightarrow 0} = 0,\rho_{T\rightarrow 0} = \textit{const.}$ 
  \textbf{(6)磁阻}($R$随$B$的变化)(O)理想:一种载流子,完美球形F面,洛伦兹力平衡于电场力,e运动将对是否有磁场不敏感,磁阻为0.(I)真实(i)$E_F$并不是严格球形,$v_{F},m^{*},\tau$各向异性;(ii)多条能带经过$E_{F}$,各能带$v_{F},m^{*},\tau$不同.[例]两能带:$\frac{\Delta\sigma}{\sigma_{0}} = -\frac{\sigma_{10}\sigma_{20}}{(\sigma_{10} + \sigma_{20})^{2}} (\omega_{c1}\tau_{1} - \omega_{c2}\tau_{2})^{2}\Rightarrow \frac{\Delta \rho}{\rho_{0}} = \frac{\rho(B) - \rho(0)}{\rho(0)}\propto B^{2}>0$\textbf{(7)相位效应}(O)与杂质弹性散射,e相干:$\vec{k}\rightarrow\vec{k}'(|\vec{k}| = |\vec{k}'|),\phi\rightarrow\phi'$;与声子非弹性散射,e非相干:$\phi = e^{-iEt/h}$.相位相干长度$l_{\phi} = v_{F}\tau_{2}$.(I)从 $x'$ 到 $x''$ 的总概率:$P=|\sum_{i}A_{i}|^{2}=|\sum_{i}A_{i}^{2}|+\sum_{i\neq j}A_{i}A_{j}$.(II)WL.环路:$P = |A_{+}|^{2} + |A_{-}|^{2} + A_{+}A_{-}^{*} + A_{+}^{*}A_{-} = 4A^{2}$,大于经典概率$P'= 2A^{2}$,$\sigma\downarrow$, $R\uparrow$.对2D:$\Delta\sigma = -\sigma_{00}\ln{\frac{\tau_{2}}{\tau_{1}}} = \sigma_{00}p\ln{T}$.(III)负磁阻:$\vec{B} = \nabla\times\vec{A},\varphi(\vec{r}) = \varphi_{0}(\vec{r}) = e^{-\frac{ie}{\hbar}\int \vec{A}(\vec{r}')\cdot\mathrm{d}\vec{r}'},A_{+}:A_{+}e^{-\frac{ie}{\hbar}\oint\vec{A}\cdot\mathrm{d}\vec{l}} = A_{+}e^{-\frac{ie}{\hbar}\iint\vec{B}\cdot\mathrm{d}\vec{S}} = A_{+}e^{-\frac{ie}{\hbar}\Phi},A_{(-)}:A_{(-)}e^{\frac{ie}{\hbar}\Phi} = A_{(-)}e^{i2\pi \Phi/(2\Phi_{0})}(\Phi_{0} = \frac{h}{2e}),P = 2A^{2}[1 + \cos^{2}{(\frac{2\pi\Phi}{\Phi_{0}})}]\leq 4A^{2}$
   \textbf{输运现象}$\vec{J}_{Q} = -\kappa\nabla T$;$\vec{J}_{Q} = -D\nabla n$;$\vec{J}_{e} = -\sigma\nabla\varphi$
  \textbf{(1)非平衡分布函数}:$f_{n}(\vec{r},\vec{k},t)\frac{\mathrm{d}\vec{r}\mathrm{d}\vec{k}}{(2\pi)^{3}}$($t$的第$n$能带中,在$(\vec{r},\vec{k})$处单位体积$\mathrm{d}\vec{r}\mathrm{d}\vec{k}$某自旋的平均e数)\textbf{(2)非平衡电流}$\vec{J}_{e} = -en(\vec{r},t)\vec{v}_{d}=-\frac{2}{(2\pi)^{3}}\int e\vec{v}_{\vec{k}}f(\vec{r},\vec{k},t)\mathrm{d}\vec{k}$\textbf{(3)平衡}:$\vec{J} = -\frac{2}{(2\pi)^{3}}\int e\vec{v}_{\vec{k}}f_{0}\mathrm{d}\vec{k} = 0$\textbf{(4)从平衡到非平衡}:$\frac{\mathrm{d}\vec{k}}{\mathrm{d}t} = -\frac{e\vec{E}}{\hbar},\vec{J} = -\frac{2}{(2\pi)^{3}}\int e\vec{v}_{\vec{k}}f\mathrm{d}\vec{k}\neq 0$\textbf{(5)玻尔兹曼方程}$\frac{\partial f}{\partial t} = (\frac{\partial f}{\partial t})_{\text{漂移}} + (\frac{\partial f}{\partial t})_{\text{碰撞}}$(I)漂移无碰撞:$f(\vec{r},\vec{k},\vec{t})=f(\vec{r}-\dot{\vec{r}}\mathrm{d}t, \vec{k} - \dot{\vec{k}}\mathrm{d}t, t - \mathrm{d}t)$(II)碰撞+漂移:$f(\vec{r},\vec{k},\vec{t}) = f(\vec{r}-\dot{\vec{r}}\mathrm{d}t, \vec{k} - \dot{\vec{k}}\mathrm{d}t, t - \mathrm{d}t) + (\frac{\partial f}{\partial t})_{\text{碰撞}}\mathrm{d}t$.稳态$(\partial_{t}f=0)\dot{\vec{k}}\frac{\partial f}{\partial\vec{k}} + \dot{\vec{r}}\frac{\partial f}{\partial\vec{r}}=(\frac{\partial f}{\partial t})_{\text{碰撞}}$.近似条件:$f = f_{0} + f_{1} (f_{1}\ll f_{0}),(\frac{\partial f}{\partial t}) = \frac{f_{0} - f}{\tau} = -\frac{f_{1}}{\tau}$.近似玻尔兹曼方程:$\dot{\vec{k}}\frac{\partial f_{0}}{\partial\vec{k}} + \dot{\vec{r}}\frac{\partial f_{0}}{\partial\vec{r}} = -\frac{f_{1}}{\tau}$(III)直流$\sigma$.仅$\vec{E}$下:$-\frac{e\vec{E}}{\hbar}\frac{\partial f_{0}}{\partial\vec{k}} = -\frac{f_{1}}{\tau},\vec{J}_{e} = -\frac{2e}{(2\pi)^{3}}\int f\vec{v}_{\vec{k}}\mathrm{d}\vec{k} = -\frac{e}{4\pi^{3}}\int(f_{0}+f_{1})\vec{v}_{\vec{k}}\mathrm{d}\vec{k}=-\frac{e}{4\pi^{3}}\int f_{1}\vec{v}_{\vec{k}}\mathrm{d}\vec{k}$.已知$f_{1} = \frac{e\tau\vec{E}}{\hbar}\frac{\partial f_{0}}{\partial\vec{k}};\frac{\partial f_{0}}{\partial\vec{k}} = \frac{\partial f_{0}}{\partial\epsilon}\frac{\partial\epsilon}{\partial\vec{k}};\vec{k} = \frac{1}{\hbar}\frac{\partial\epsilon}{\partial\vec{k}}$,$\vec{J}_{e} = \frac{e^{2}}{4\pi^{3}}\int\tau\frac{\partial f_{0}}{\partial\epsilon}\vec{k}_{\vec{k}}(\vec{v}_{\vec{k}}\cdot\vec{E})\mathrm{d}\vec{k}=\frac{e^{2}}{4\pi^{3}}\int\tau\frac{\vec{v}_{\vec{k}}(\vec{v}_{\vec{k}}\cdot\vec{E})}{\hbar v_{k}}(-\frac{\partial f_{0}}{\partial\epsilon})\mathrm{d}S\mathrm{d}\epsilon= \frac{e^{2}}{4\pi^{3}\hbar}\int\tau\frac{\vec{v}_{k}(\vec{v}_{\vec{k}}\cdot\vec{E})}{v_{k}}\mathrm{d}S_{F},\vec{J}_{e} = [\frac{e^{2}}{4\pi^{3}\hbar}\int\tau\frac{\vec{v}_{\vec{k}}\vec{v}_{\vec{k}}}{v_{k}}\mathrm{d}S_{F}]\cdot\vec{E}=\overleftrightarrow{\sigma}\cdot\vec{E}$[例]立方晶系:$\sigma=\sigma_{xx}=\sigma_{yy}=\sigma_{zz}=\frac{\left(\sigma_{xx}+\sigma_{yy}+\sigma_{zz}\right)}{3}$=$\frac{e^{2}}{4\pi^{3}\hbar}\int\tau\frac{v_{k_{x}}^{2}}{v_{k}}\mathrm{d}S_{F}=\frac{1}{12\pi^{3}}\frac{e^{2}}{\hbar}\int\tau v_{k}\mathrm{d}S_{F}$.各向同性$m^{*},\tau:\sigma = \frac{\tau e^{2}}{12\pi^{3}m^{*}}\int k_{F}\mathrm{d}S_{F}$.球面F面:$\sigma=\frac{\tau}{3\pi^{2}}\frac{e^{2}k_{F}^{3}}{m^{*}}=\frac{ne^{2}\tau_{E_{F}}}{m^{*}}$\textbf{(6)热电势}.$T,\mu$:$\frac{\partial f_{0}}{\partial T}=-\frac{\partial f_{0}}{\partial\epsilon}\frac{\epsilon-\mu}{T},\frac{\partial f_{0}}{\partial\mu} = -\frac{\partial f_{0}}{\partial\epsilon}$,无$\vec{E}$方程:$-\frac{\partial f_{0}}{\partial\epsilon}\vec{v}_{\vec{k}}\cdot[\frac{\epsilon_{k}-\mu}{T}\nabla T + \nabla\mu] = -\frac{f_{1}}{\tau}$.电流密度:$\vec{J}_{e}=\frac{e}{4\pi^{3}}\int\tau\frac{(\vec{v}_{\vec{k}}\vec{v}_{\vec{k}})\cdot\nabla\mu}{\hbar v_{k}}(-\frac{\partial f_{0}}{\partial\epsilon})\mathrm{d}S\mathrm{d}\epsilon + \frac{e}{4\pi^{3}}\int\tau\frac{(\vec{v}_{\vec{k}}\vec{v}_{\vec{k}})\cdot\nabla T}{\hbar v_{k}}(\frac{\epsilon - \mu}{T})(-\frac{\partial f_{0}}{\partial\epsilon})\mathrm{d}S\mathrm{d}\epsilon$.$\mu$梯度$\frac{\nabla\mu}{e}$与外$\vec{E}$等价.
  \textbf{(7)热流}类比$\vec{J}_{e} = \frac{e^{2}}{4\pi^{3}}\int\tau\frac{(\vec{v}_{\vec{k}}\vec{v}_{\vec{k}})\cdot(\vec{E} + \frac{\nabla\mu}{e})}{\hbar v_{k}}(-\frac{\partial f_{0}}{\partial\epsilon})\mathrm{d}S\mathrm{d}\epsilon + \frac{e}{4\pi^{3}}\int\tau\frac{(\vec{v}_{\vec{k}}\vec{v}_{\vec{k}})\cdot\nabla T}{\hbar v_{k}}(\frac{\epsilon-\mu}{T})(-\frac{\partial f_{0}}{\partial\epsilon})\mathrm{d}S\mathrm{d}\epsilon$,定义热流:$\vec{J}_{Q} = \frac{1}{4\pi^{3}}\int(\epsilon_{k} - \mu)\vec{v}_{k} f_{1}\mathrm{d}\vec{k}=-\frac{e}{4\pi^{3}}\int\vec{E}\cdot[\tau\frac{(\vec{v}_{\vec{k}}\vec{v}_{\vec{k}})(\epsilon_{k} - \mu)}{\hbar v_{k}}(-\frac{\partial f_{0}}{\partial\epsilon})\mathrm{d}S\mathrm{d}\epsilon] - \frac{1}{4\pi^{3}}\int\nabla T\cdot[\tau\frac{(\vec{v}_{\vec{k}}\vec{v}_{\vec{k}})}{\hbar v_{k}}\frac{(\epsilon - \mu)^{2}}{T}(-\frac{\partial f_{0}}{\partial\epsilon})\mathrm{d}S\mathrm{d}\epsilon]$.设$\zeta_{n} = \frac{\tau}{12\pi^{3}\hbar}\int v_{k}(\epsilon_{k} - \mu)^{n}\left(-\frac{\partial f_{0}}{\partial\epsilon}\right)\mathrm{d}S\mathrm{d}\epsilon,\vec{J}_{e} = e^{2}\zeta_{0}\vec{E} - \frac{e}{T}\zeta_{1}(-\nabla T),\vec{J}_{Q} = -e\zeta_{1}\vec{E} + \frac{1}{T}\zeta_{2}(-\nabla T)$.无外$\vec{E}:\vec{J}_{e} = 0\Rightarrow e^{2}\zeta_{0}\vec{E} - \frac{e}{T}\zeta_{1}(-\nabla T) = 0,\vec{E} = \frac{1}{eT}\frac{\zeta_{1}}{\zeta_{0}}(-\nabla T)$,热流密度:$\vec{J}_{Q} = \frac{1}{T}\left(\zeta_{2} - \frac{\zeta_{1}^{2}}{\zeta_{0}}\right)(-\nabla T)$,热导率:$\kappa = \frac{1}{T}\left(\zeta_{2} - \frac{\zeta_{1}^{2}}{\zeta_{0}}\right),\sigma = e^{2}\zeta_{0}$.\textbf{(8)补充:热电势}热电场:$\vec{E} = \frac{1}{eT}\frac{\zeta_{1}}{\zeta_{0}}(-\nabla T)$,热电系数(单位$T$差下材料中$\phi$的变化量):$S = -\frac{1}{eT}\frac{\zeta_{1}}{\zeta_{0}} = -\frac{\pi^{3}}{3}\frac{k_{B}^{2}T}{e}\left[\frac{\partial\ln{\sigma}}{\partial\epsilon}\right]_{E_{F}}\propto T\left(\frac{\partial\ln{\langle\tau\rangle}}{\partial\epsilon} + \frac{\partial\ln{\langle v_{k}\rangle}}{\partial\epsilon} + \frac{\partial\ln{S}}{\partial\epsilon}\right)_{E_{F}}$
  \textbf{多e}\textbf{(0)原始:}$\hat{H}_{T} = \sum_{i}\frac{|\vec{p_{i}}|^2}{2m}+\sum_{n}\frac{|\vec{p}_{n}|^2}{2M_{n}}+\frac{1}{2}\sum_{ij}'\frac{e^2}{|\vec{r}_{i}-\vec{r_{j}}|} + \frac{1}{2}\sum_{nn'}'\frac{Z_{n}Z_{n'}e^{2}}{|\vec{R}_{n}-\vec{R}_{n'}|} +\sum_{n,i}V_{n}(\vec{r}_{i}-\vec{R}_{n})+\hat{H}_{R}$(价e动,原子实动,e间库伦,原子实间库伦,e和原子实之间,s-L修正)\textbf{(1)B-O绝热}.(I)e:$\hat{H}_{e} = \sum_{i}[\frac{|\vec{p_{i}}|^2}{2m}+\sum_{n}V_{n}(\vec{r_{i}}-\vec{R_{n}})] + \frac{1}{2}\sum_{ij}'\frac{e^{2}}{|\vec{r_{i}}-\vec{r_{j}}|} + \hat{H}_{R}$,原子实:$\hat{H}_{c} = \sum_{n}\frac{|\vec{p_{n}}|^2}{2M_{n}} + \frac{1}{2} \sum_{nn'}'\frac{Z_{n}Z_{n'}e^2}{|\vec{R_{n}}-\vec{R_{n'}}|} + V_{\text{ec}}(\{\vec{R_{n}}\})$(II)$\{-\frac{\hbar^{2}}{2m}\sum_{j}\nabla_{j}^{2}-\sum_{j,l}\frac{Z_{l}e^{2}}{|r_{j}-R_{l}|}+\frac{1}{2}\sum_{j\neq j'}\frac{e^{2}}{|r_{j}-r_{j'}|} - E\}\Psi(\{r_{N}\}) = 0,\hat{P}_{jj'}\Psi = -\Psi,.n(r) = n(r;\{R_{\mathcal{N}}\}),E = E(\{R_{\mathcal{N}}\})$.\textbf{(2)}$H_{2}$\textbf{Model}:(I)HL:$\Psi_{HL}=A[\varphi_{H}(r_{1}-R_{1})\varphi_{H}(r_{2}-R_{2})+\varphi_{H}(r_{1}-R_{2})\varphi_{H}(r_{2}-R_{1})]\chi_{0}$(HL = Heitler-London).$\varphi_{H}(r)$ 是e轨道在基态; $\chi_{0}$ 代表自旋单子态.(II)Mullikan Ansatz:$\Psi_{\text{HF}} = \frac{1}{\sqrt{2}}\mathrm{Det}[\varphi_{m}(r_{1})\alpha(1)\varphi_{m}(r_{2})\beta(2)]$.(III)JC:$\Psi_{JC} = \Psi(r_{1},r_{2})\chi_{0}$\textbf{(III)Hartree-Fock}对e:$\hat{H} = -\sum_{i}\frac{\hbar^2}{2m_{e}}\nabla_{\vec{r_{i}}}^{2} + \sum_{i}V_{\text{ion}}(\vec{r_{i}}) + \frac{e^2}{2}\sum_{(i\neq j)}\frac{1}{|\vec{r_{i}}-\vec{r_{j}}|}$,多体态:$\Psi^{H}(\{\vec{r}_{i}\}) = \phi_{1}(\vec{r}_{1})\dots\phi_{N}(\vec{r}_{N})$, $E^{H} = \langle\Psi^{H}|\hat{H}|\Psi^{H}\rangle = \sum_{i}\langle\phi_{i}|\frac{-\hbar^{2}\nabla_{\vec{r}}^2}{2m_{e}} + V_{\text{ion}}(\vec{r})|\phi_{i}\rangle + \frac{e^{2}}{2}\sum_{ij(i\neq j)}\langle\phi_{i}\phi_{j}|\frac{1}{|\vec{r}-\vec{r}'|}|\phi_{i}\phi_{j}\rangle,\delta[E^{H}- \sum_{i}\epsilon_{i}(\langle\phi_{i}|\phi_{i}\rangle-1)] = 0,\langle\delta\phi_{i}|-\frac{\hbar^2\nabla_{\vec{r}}^2}{2m_{e}}+V_{\text{ion}}(\vec{r})|\phi_{i}\rangle + e^{2}\sum_{i\neq j}\langle\delta\phi_{i}\phi_{j}|\frac{1}{|\vec{r}-\vec{r}'|}|\phi_{i}\phi_{j}\rangle - \epsilon_{i}\langle\delta\phi_{i}|\phi_{i}\rangle=\langle\delta\phi_{i}|[-\frac{\hbar^2\nabla_{\vec{r}}^2}{2m_{e}}+V_{\text{ion}}+e^2\sum_{i\neq j}\langle\phi_{j}|\frac{1}{|\vec{r}-\vec{r}'|}|\phi_{j}\rangle-\epsilon]|\phi_{i}\rangle = 0$.Hatree:$[-\frac{\hbar^2\nabla{\vec{r}}^{2}}{2m_{e}}+V_{\text{ion}}(\vec{r})+e^2\sum_{j\neq i}\langle\phi_{j}|\frac{1}{|\vec{r}-\vec{r}'|}|\phi_{i}\rangle]\phi_{i}(\vec{r}) = \epsilon_{i}\phi_{i}(\vec{r})$,Hatree势:$V_{i}^{H}(\vec{r}) = e^2\sum_{i\neq j}\langle\phi_{j}|\frac{1}{|\vec{r}-\vec{r}'|}|\phi_{j}\rangle$.平均场近似:Hatree-Fock多体态:$\Psi^{\text{HF}}(\{\vec{r}_{i}\}) = \frac{1}{\sqrt{N!}}\left|\begin{array}{ccc} 
    \phi_{1}(\vec{r}_{1})   & \phi_{1}(\vec{r}_{N}) \\
    \phi_{N}(\vec{r}_{1})   & \phi_{N}(\vec{r}_{N}) \\
\end{array}\right|.(\phi_{i}(\vec{r}) \approx \psi_{i}(\vec{r})\chi_{i}(\sigma))$,总能量:$E^{\text{HF}} = \langle\Psi^{\text{HF}}|\hat{H}|\Psi^{\text{HF}}\rangle=\sum_{i}\langle\phi_{i}|\frac{-\hbar^2\nabla_{\vec{r}}^{2}}{2m_{e}}+V_{\text{ion}}(\vec{r})|\phi_{i}\rangle + \frac{e^{2}}{2}\sum_{ij(i\neq j)}\langle\phi_{i}\phi_{j}|\frac{1}{|\vec{r}-\vec{r}'|}|\phi_{i}\phi_{j}\rangle - \frac{e^2}{2}\sum_{ij(i\neq j)}\langle\phi_{i}\phi_{j}|\frac{1}{|\vec{r}-\vec{r}'|}|\phi_{j}\phi_{i}\rangle$,$[\frac{-\hbar^2\nabla_{\vec{r}}^{2}}{2m_{e}}+V_{\text{ion}}+V_{i}^{H}(\vec{r})]\phi_{i}(\vec{r}) - e^2\sum_{j\neq i}\langle\phi_{j}|\frac{1}{|\vec{r}-\vec{r}'|}|\phi_{i}\rangle\phi_{j}(\vec{r})=\epsilon_{i}\phi_{i}(\vec{r})$.密度:$\rho_{i}(\vec{r}) = |\phi_{i}(\vec{r})|^2,\rho(\vec{r}) = \sum_{i}\rho_{i}(\vec{r});V_{i}^{H}(\vec{r}) = e^{2}\sum_{j\neq i}\int\frac{\rho_{j}(\vec{r}')}{|\vec{r}-\vec{r}'|}\mathrm{d}\vec{r}' = e^{2}\int\frac{\rho(\vec{r}')-\rho_{i}(\vec{r}')}{|\vec{r}-\vec{r}'|}\mathrm{d}\vec{r}';$\textbf{单粒子交换:}$\rho_{i}^{X}(\vec{r},\vec{r}') = \sum_{j\neq i}\frac{\phi_{i}(\vec{r}')\phi_{i}^{*}(\vec{r})\phi_{j}(\vec{r})\phi_{j}^{*}(\vec{r}')}{\phi_{i}(\vec{r})\phi_{i}^{*}(\vec{r})}$;HF势:$V_{i}^{HF}(\vec{r}) = e^{2}\int\frac{\rho(\vec{r}')}{|\vec{r}-\vec{r}'|}\mathrm{d}\vec{r}'-e^2\int\frac{\rho_{i}(\vec{r}')+\rho_{i}^{X}(\vec{r},\vec{r}')}{|\vec{r}-\vec{r}'|}\mathrm{d}\vec{r}',\text{单HF:}[\frac{-\hbar^2\nabla_{\vec{r}}^{2}}{2m_{e}}+V_{ion}(\vec{r})+V_{i}^{HF}(\vec{r})]\phi_{i}(\vec{r}) = \epsilon_{i}\phi_{i}(\vec{r})$.\textbf{(IV)Jellium Model(均匀电子气)}$\phi_{i}(\vec{r}) = \frac{e^{i\vec{k}_{i}\cdot\vec{r}}}{\sqrt{\Omega}}$($\Omega$为晶胞体积.均匀电子气的波矢的数值范围为$k\in[0,k_{F}]$).$\frac{4\pi}{3}r_{s} = \frac{\Omega}{N} = n^{-1} = \frac{3\pi^2}{k_{F}^{3}},\frac{\hbar^{2}}{2m_{e}a_{0}^{2}} = \frac{e^{2}}{2a_{0}} = 1\mathrm{Ry}$.态方程$[-\frac{\hbar^{2}\nabla_{\vec{r}}^{2}}{2m_{e}}-e^{2}\int\frac{\rho_{\vec{k}}^{HF}(\vec{r},\vec{r}')}{|\vec{r}-\vec{r}'|}\mathrm{d}\vec{r}']\phi_{\vec{k}}(\vec{r}) = \epsilon_{\vec{k}}\phi_{\vec{k}}(\vec{r})$.平面波证明:$-\frac{\hbar^{2}\nabla^{2}}{2m_{e}}\frac{e^{i\vec{k}\cdot\vec{r}}}{\sqrt{\Omega}} = \frac{\hbar^{2}\hat{k}^{2}}{2m_{e}}\frac{e^{i\vec{k}\cdot\vec{r}}}{\sqrt{\Omega}},e^{2}[\int\frac{\rho_{\vec{k}}^{HF}(\vec{r},\vec{r}')}{|\vec{r}-\vec{r}'|}\mathrm{d}\vec{r}']\phi_{\vec{k}}(\vec{r}) = \frac{-e^{2}}{\sqrt{\Omega}}\int\frac{\rho_{\vec{k}}^{HF}(\vec{r},\vec{r}')}{|\vec{r}-\vec{r}'|}\mathrm{d}\vec{r}'e^{i\vec{k}\cdot\vec{r}}=\frac{-e^{2}}{\sqrt{\Omega}}\sum_{\vec{k}'}\int\frac{\phi_{\vec{k}}(\vec{r}')\phi_{\vec{k}}^{*}(\vec{r})\phi_{\vec{k}'}(\vec{r})\phi_{\vec{k}'}^{*}(\vec{r}')}{\phi_{\vec{k}}(\vec{r})\phi_{\vec{k}}^{*}(\vec{r})}\frac{1}{|\vec{r}-\vec{r}'|}\mathrm{d}\vec{r}'e^{i\vec{k}\cdot\vec{r}}=\frac{-e^{2}}{\sqrt{\Omega}}\sum_{\vec{k}'}\int\frac{e^{-i(\vec{k}-\vec{k}')\cdot(\vec{r}-\vec{r}')}}{\Omega}\frac{\mathrm{d}\vec{r}'e^{i\vec{k}\cdot\vec{r}}}{|\vec{r}-\vec{r}'|}$,$(\int\frac{1}{r}e^{i\vec{k}\cdot\vec{r}}\mathrm{d}\vec{r} = \frac{4\pi^2}{k^{2}}$,$\sum_{\vec{k}}f(\vec{k}) = \frac{\Omega}{(2\pi)^{3}}\int f(\vec{k})\mathrm{d}\vec{k})$,$\frac{-4\pi^{2}}{\sqrt{\Omega}}[\int_{k'<k_{F}}\frac{\mathrm{d}\vec{k}'}{(2\pi)^3}\frac{1}{|\vec{k}-\vec{k}'|^{2}}]e^{i\vec{k}\cdot\vec{r}} = -\frac{e^{2}}{\pi}k_{F}F(\frac{k}{k_{F}})\frac{e^{i\vec{k}\cdot\vec{r}}}{\sqrt{\Omega}}.(F(x) = 1 + \frac{1 - x^2}{2x}\ln{|\frac{1+x}{1-x}|}).\text{(II)}\phi_{\vec{k}}(\vec{r})$能量:$\epsilon_{\vec{k}} = \frac{\hbar^{2}k^{2}}{2m_{e}}-\frac{e^{2}}{\pi}k_{F}F(\frac{k}{k_{F}})$,总能量$E^{HF} = 2\sum_{k<k_{F}}\frac{\hbar^{2}|\vec{k}|^{2}}{2m_{e}} - \frac{e^{2}k_{F}^{2}}{\pi}\sum_{k<k_{F}}[1+ \frac{k_{F}^{2}-k^{2}}{2kk_{F}}\ln{|\frac{k_{F}+k}{k_{F}-k}|}]$.平均能:$\frac{E^{HF}}{N} = \frac{3}{5}\epsilon_{F} -  \frac{3}{4}\frac{e^{2}k_{F}}{\pi}=[\frac{2.21}{(r_{s}/a_{0})^{2}}-\frac{0.916}{(r_{s}/a_{0})}]\mathrm{Ry}$.交换能:$\frac{E^{X}}{N} = -\frac{3e^{2}}{4}(\frac{3}{\pi})^{\frac{1}{3}}n^{\frac{1}{3}} = -1.447(a_{0}^{3}n)^{\frac{1}{3}}\mathrm{Ry}$;高$n:\frac{E}{N} = [\frac{2.21}{(r_{s}/a_{0})^{2}}-\frac{0.916}{(r_{s}/a_{0})}+0.0622\ln{\frac{r_{s}}{a_{0}}}-0.096+\mathcal{O}(\frac{r_{s}}{a_{0}})]$\textbf{DFT}(1)$\mathcal{H} =T+W+V= -\sum_{i}\frac{\hbar^{2}}{2m_{e}}\nabla_{\vec{r}_{i}}^{2} + \sum_{i}V_{\text{ion}}(\vec{r}_{i}) + \frac{e^{2}}{2}\sum_{ij(j\neq i)}\frac{1}{|\vec{r}_{i}-\vec{r}_{j}|}$.DF:$F[n(r)] = \langle\Psi|T + W|\Psi\rangle=F[n(\vec{r})] = T^{S}[n(\vec{r})] + \frac{e^{2}}{2}\iint\frac{n(\vec{r})n(\vec{r}')}{|\vec{r}-\vec{r}'|}\mathrm{d}\vec{r}\mathrm{d}\vec{r}' + E^{XC}[n(\vec{r})],E[n(\vec{r})] = \langle\Psi|\mathcal{H}|\Psi\rangle = F[n(\vec{r})] + \int V(\vec{r})n(\vec{r})\mathrm{d}\vec{r}$,变分:$\delta n(\vec{r}) = \delta\phi_{i}^{\*}(\vec{r})\phi_{i}(\vec{r})$,约束$\int\delta n(\vec{r})\mathrm{d}\vec{r} = \int\delta\phi_{i}^{\*}(\vec{r})\phi_{i}(\vec{r})\mathrm{d\vec{r}} = 0$,Kohn-Sham:$[-\frac{\hbar^{2}}{2m_{e}}\nabla_{\vec{r}}^{2}+V^{\text{eff}}(\vec{r},n(\vec{r}))]\phi_{i}(\vec{r}) = \epsilon_{i}\phi_{i}(\vec{r})(V^{\text{eff}}(\vec{r},n(\vec{r})) = V(\vec{r}) + e^{2}\int\frac{n(\vec{r}')}{|\vec{r}-\vec{r}'|}\mathrm{d}\vec{r}' + \frac{\delta E^{XC}[n(\vec{r})]}{\delta n(\vec{r})})$.$E^{XC}[n(\vec{r})] = \int n(\vec{r})\epsilon^{XC}([n],\vec{r})\mathrm{d}\vec{r}$.LDA:$E^{XC}_{\text{LDA}} = \int\epsilon^{XC}[n(\vec{r})]n(\vec{r})\mathrm{d}\vec{r}$;GGA:$E^{XC}_{\text{GGA}} = \int\epsilon^{XC}[n(\vec{r}),|\nabla n(\vec{r})|]n(\vec{r})\mathrm{d}\vec{r}$.
\textbf{Kittle(1)Wannier}正交:$\Psi_{k}=N^{-\frac{1}{2}}e^{ik\cdot r}u_{k}(\vec{r}),\int\mathrm{d}\tau w^{*}(\vec{r}-\vec{r}_{n})w(\vec{r}-\vec{r}_{n})=\frac{1}{N}\sum_{kk'}e^{ikr_{n}}e^{-ik'r_{m}}\delta_{kk'}=\delta_{nm}$.线晶格:$\psi_{k}=N^{-\frac{1}{2}}e^{ikx}u_{0}(x),w(x-x_{n})=N^{-1}u_{0}(x)\sum e^{ik(x-x_{n})}=u_{0}(x)\frac{1}{N}\frac{L}{2\pi}\int_{-\pi/a}^{\pi/a}e^{ik(x-x_{m})}\mathrm{d}k=u_{0}(x)\frac{\sin{\frac{\pi}{a}(x-x_{n})}}{\frac{\pi}{a}(x-x_{n})}$.\textbf{(2)2D矩形}$\vec{a}=2A\hat{x},\vec{b}=4A\hat{y}.\vec{A}=\frac{\pi}{a}\hat{x},\vec{B}=\frac{\pi}{2A}\hat{y}.\vec{G}=\frac{2\pi}{4A}(2g_{1}\hat{x}+g_{2}\hat{y}).2\frac{\pi k_{F}^{2}}{\frac{(2\pi)^{2}}{ab}}=Z,k_{F}=\sqrt{\frac{\pi}{4A^{2}}}$\textbf{(3)3D六角密}$a,c.U(\vec{r})=\sum_{n=1}^{m}U(\vec{r}-\vec{R}_{n})=\sum_{n=1}^{m}\sum_{G}U(G)e^{iG(r-R_{n})}=\sum_{G}S(\vec{G})e^{iGr}.G=m_{1}\vec{b}_{1}+m_{2}\vec{b}_{2}+m_{3}\vec{b}_{3}.a_{1}=a(1,0,0),a_{2}=a(-\frac{1}{2},\frac{\sqrt{3}}{2},0),a_{3}=c(0,0,1);b_{1}=\frac{2\pi}{a}(1,\frac{1}{\sqrt{3}},0),b_{2}=\frac{2\pi}{a}(0,\frac{2}{\sqrt{3}},0),b_{3}=\frac{2\pi}{c}(0,0,1).\vec{G}=2\pi[\frac{m_{1}}{a}\hat{x}+(-\frac{m_{1}}{\sqrt{3}a}+\frac{m_{2}}{\sqrt{3}a})\hat{y}+\frac{m_{3}}{c}\hat{z}].S(\vec{G})=1+e^{-i2\pi}(\frac{2m_{1}}{3}+\frac{m_{2}}{3}+\frac{m_{3}}{2})(i)\vec{G}_{c}(m_{1},m_{2}=0,m_{3}=\pm 1)S(G)=0,U(\pm\vec{G}_{c}=0)(ii)\vec{G}=\pm 2\vec{G}_{c}S(\pm 2\vec{G}_{c})=2,U(\pm 2\vec{G}_{c}\neq 0).U(\pm\vec{G}_{c})=0$,能区$\uparrow$,容$4N$e.\textbf{(4)开轨道}$d_{\text{BZ}}=G,B\perp\partial\text{BZ}.\frac{\hbar\mathrm{d}\mathbf{k}}{\mathrm{d}t}=\frac{-e\mathbf{v}\times\mathbf{B}}{c}.T=\int\mathrm{d}t=\int_{0}^{G}\frac{c\hbar}{evB}\mathrm{d}k=\frac{c\hbar G}{evB}.\text{if }\vec{B}=B\hat{z},\frac{\mathrm{d}U_{x}}{\mathrm{d}t}=\frac{-ev_{y}B}{mc},\frac{\mathrm{d}U_{y}}{\mathrm{d}t}=\frac{ev_{x}B}{mc},\frac{\mathrm{d}U_{z}}{\mathrm{d}t}=0.\textbf{(5)K的De Hass-van Alphen}\Delta(\frac{1}{B})=\frac{2\pi e}{hcS},S=\pi k_{F}^{2},k_{F}=(\frac{3\pi^{2}N}{V}).n=\frac{2}{a^{3}},\Delta(\frac{1}{B})=5.46\times 10^{-9}\text{G}^{-1}.\text{极值轨道}A_{n}=(\frac{\hbar c}{eB})^{2}S_{n}(S_{n}=\pi k_{F}^{2})\textbf{(6)开轨道磁致}\mathbf{R}.\sigma_{yy}=S\sigma_{0}.\text{if }\omega_{c}\tau\ll 1.\overleftrightarrow{\sigma}=[{Q^{-2},Q^{-1},0},{Q^{-1},S,0},{0,0,1}](Q=\omega_{c}\tau).$霍尔:$j_{y}=\sigma_{yx}E_{x}+\sigma_{yy}E_{y}=0,E_{y}=\frac{-\sigma_{yx}E_{x}}{\sigma_{yy}}=\frac{-E_{x}}{SQ}.j_{x}=\sigma_{xx}E_{x}+\sigma_{xy}E_{y}=\frac{\sigma_{0}(S+1)E_{x}}{SQ^{2}},\rho=\frac{E_{x}}{j_{x}}=\frac{SQ^{2}}{\sigma_{0}(S+1)}$\textbf{Yan(1)周期势}$V(x)=\{^{\frac{m\omega^{2}}{2}[b^{2}-(x-na)^{2}],[na-b,na+b]}_{0,[(n-1)a+b,na-b]}.\overline{V}=\frac{1}{4b}\int_{-b}^{b}\frac{m\omega^{2}}{2}(b^{2}-x^{2})\mathrm{d}x=\frac{b^{2}m\omega^{2}}{6}.V_{n}=\frac{1}{a}\int_{-b}^{b}e^{\frac{-i2\pi nt}{a}}V(t)\mathrm{d}t=\frac{1}{a}\int_{-b}^{b}\frac{m\omega^{2}}{2}(b^{2}-t^{2})e^{\frac{-i2\pi nt}{a}}\mathrm{d}t=\frac{m\omega^{2}b^{2}}{2\pi^{3}n^{3}}(8\sin{\frac{n\pi}{2}}-4n\pi\cos{\frac{n\pi}{2}})$.$|2V_{1}|=\frac{8m\omega^{2}b^{2}}{\pi^{3}},|2V_{2}|=\frac{m\omega^{2}b^{2}}{\pi^{2}}.\textbf{(2)紧束缚}.(bcc):a(\frac{\pm 1}{2},\frac{\pm 1}{2},\frac{\pm 1}{2});(fcc)a(\frac{\pm 1}{2},\frac{\pm 1}{2},0),a(\frac{\pm 1}{2},0,\frac{\pm 1}{2}),a(0,\frac{\pm 1}{2},\frac{\pm 1}{2}).$\textbf{(3)1D复式晶格}.原胞长$a$,内相对距$b$.s:$\phi(r)=\frac{1}{\sqrt{N}}\sum_{R_{m}}[e{ikR_{m}^{A}}\rho(r-R_{m}^{A})+C_{B}e^{ikR_{m}^{B}}\rho(r-R_{m}^{B})].E(k)=E-J_{0}-\sum e^{-ikR_{m}}J(R_{m})=E-J_{0}-(J_{1}+e^{-ika}J_{2})e^{-ik[(n-1)a+b]}=E-J_{0}-2J_{1}\cos{\frac{ka}{2}}e^{-ik[(n-\frac{1}{2})a+b]}.$\textbf{Cu中Zn}.F球相切$\partial$BZ:$N=\frac{Vk_{F}^{3}}{3\pi{2}}=\frac{V\mathbf{\sqrt{3}\pi}}{a^{3}};\frac{2x+1}{1+x}=\frac{\sqrt{3}\pi}{4}(x=n(Zn):n(Cu))$\textbf{(4)椭色散}$\epsilon(\vec{k})=\hbar^{2}\sum_{i}^{3}\frac{k_{i}^{2}}{2m_{i}}.N=2(\frac{L}{2\pi})^{3}\frac{4}{3}(\frac{\epsilon}{\hbar^{2}})^{\frac{3}{2}}(m_{x}m_{y}m_{z})^{\frac{1}{2}}.D(\epsilon)=\frac{\mathrm{d}N}{\mathrm{d}\epsilon}=\frac{V}{2\pi^{2}}(\frac{2}{\hbar^{2}})^{\frac{3}{2}}(m_{x}m_{y}m_{z})^{\frac{1}{2}}\epsilon^{\frac{1}{2}}.U-U_0=\frac{\pi^{2}}{6}g(\epsilon)(k_{B}T)^{2},\frac{\partial U}{\partial T}=\frac{\pi^2}{3}g(\epsilon)k_{B}T^{2}$,自由:$C_V=\frac{\pi^{2}k_{B}^{2}T}{3}\frac{V}{2\pi^{2}}(\frac{2m}{\hbar^{2}})^{\frac{3}{2}}\epsilon_{F}^{\frac{1}{2}};$椭球:$C_V=\frac{\pi^{2}k_{B}^{2}T}{3}\frac{V}{2\pi^{2}}(\frac{2}{\hbar^{2}})^{\frac{3}{2}}(m_{x}m_{y}m_{z})^{\frac{1}{2}}\epsilon_{F}^{\frac{1}{2}}.$\textbf{(4)1D导体}$\epsilon(k)=\epsilon_{0}-\frac{\Delta}{2}\cos{ka}.T=0,k_{F}.v=\frac{\mathrm{d}\epsilon}{\mathrm{d}k}=\frac{a\Delta}{2\hbar}\sin{ka}.z=\int v\mathrm{d}t=\frac{\Delta}{2eE}(\cos{ka}-1),k=k_{0}-\frac{eEt}{\hbar},v=\frac{\mathrm{d}Z}{\mathrm{d}t}=\frac{\Delta a}{2\hbar}\sin{k_{0}a-\frac{eEat}{\hbar}}.v_{max}=\frac{\Delta a}{2\hbar},\omega_{\text{BO}}=\frac{eEa}{\hbar}.$$E_{F}$以下均占,以上均空:$j=\frac{-e}{L}\int_{-k_{F}}^{k_{F}}v(t)\frac{2\mathrm{d}k}{\frac{2\pi}{L}}=\frac{e\Delta}{\pi\hbar}\sin{k_{F}a}\sin{\frac{eEat}{\hbar}}.N=\frac{2k_{F}}{\frac{2\pi}{L}}2,j_{max}=\frac{e\Delta}{\pi\hbar}\sin{k_{F}a}=\frac{\Delta e}{\pi\hbar}\sin{\frac{\pi na}{2}}.$\textbf{(5)非抛(i)}$\frac{\hbar^{2}k^{2}}{2m_{e}}=\epsilon(1+\alpha\epsilon).g(\epsilon)=\frac{2}{(2\pi)^{3}}\int\frac{\mathrm{d}S_{\epsilon}}{|\nabla_{k}E(k)|}.\frac{\hbar^{2}|k|}{m}=\nabla_{k}\epsilon(1+2\alpha\epsilon),g(\epsilon)=\frac{2}{(w\pi^{3})}\int\frac{\mathrm{d}S}{\hbar^{2}|k|}m(1+2\alpha\epsilon)=\frac{4\pi k^{2}2}{(2\pi)^{3}}\frac{m(1+2\alpha\epsilon)}{\hbar^{2}k},k=\frac{\sqrt{2m\epsilon(1+\alpha\epsilon)}}{\hbar},g(\epsilon)=\frac{m(1+2\alpha\epsilon)}{\pi^{2}\hbar^{3}}\sqrt{2m\epsilon(1+\alpha\epsilon)}.$\textbf{(5)}$\mathbf{\mu(T)}.E_{g}=0.5\text{eV}.m_{h}^{*}=2m_{e}^{*},\mu=\frac{E_{c}+E_{V}}{2}+\frac{3k_{B}T}{4}\ln{\frac{m_{h}^{*}}{m_{e}^{*}}}.$设价带顶0,$\mu=0.25\text{eV}+\frac{3k_{B}T_{1}}{4}\ln{\frac{2m_{e}}{m_{e}}},\Delta\mu=\frac{3k_{B}\Delta T}{4}\ln{2}.$\textbf{(6)测带隙}$n=p\propto T^{\frac{3}{2}}e^{\frac{-E_{g}}{2k_{B}T}},\sigma\propto\frac{1}{U},\ln{U}=C+\frac{E_{g}}{2k_{B}T}$\textbf{(7)有效质量}$E_{1,2}(\vec{k})=E_{V}-\frac{\hbar^{2}}{2m}\{AK^{2}\pm [B^{2}k^{4}+C^{2}(k_{x}^{2}k_{y}^{2}+k_{y}^{2}k_{z}^{2}+k_{z}^{2}k_{x}^{2})]\}.(i)[1,0,0]:k_x=k,k_{y}=k_{z}=0.E(\vec{k})=E_{V}-\frac{\hbar^{2}}{2m}(A\pm B)k_{x}^{2},\frac{1}{m^{*}}=\frac{-\mathrm{d}^{2}E}{\hbar^{2}\mathrm{d}k^{2}}=\frac{A\pm B}{m};(ii)\forall i,k_{i}=\frac{k}{\sqrt{3}}.E=E_{V}-\frac{\hbar^{2}}{2m}\{k^{2}(A^{2}\pm (B^{2}+\frac{C^{2}}{3})^{\frac{1}{2}})\},m^{*}=\frac{m}{A^{2}\pm(B^{2}+\frac{C^{2}}{3})^{\frac{1}{3}}}$\textbf{(8)杂质轨道}$E_{g}=0.23\text{eV},\epsilon=18,m_{e}^{*}=0.015m$.电离$E_{d}=\frac{e^{4}m_{e}}{2\epsilon^{2}\hbar^{2}}$.基态半径$a_{d}=\frac{\epsilon\hbar^{2}}{mne^{2}}$.杂质重叠临界浓度(fcc):$V=Na^{3},N_{0}=\frac{3V}{4\pi a_{d}^{3}},C_{\text{min}}=\frac{N_{0}}{4N}$.\textbf{(9)}$\mathbf{\sigma_{\text{min}}}.\sigma=ne\mu_{n}+pe\mu_{p},np=n_{i}^{2},\sigma=\frac{n_{i}^{2}e\mu_{n}}{p}+pe\mu_{p}.$Min:$p=\sqrt{\frac{\mu_{n}}{\mu_{p}}}n_{i},n=\sqrt{\frac{\mu_{p}}{\mu_{n}}}n_{i}.\sigma=2n_{i}e\sqrt{\mu_{n}+\mu_{p}},\sigma_{i}=n_{i}e(\mu_{n}+\mu_{p}),\sigma=\frac{2\sigma_{i}\sqrt{\mu_{n}\mu_{p}}}{\mu_{n}+\mu_{p}}$.\textbf{(10)施主电离}.施主密度$n=10^{13}\text{cm}^{-3}$,电离$E_{d}=1\text{meV},m^{*}=0.01m.$低温$(k_{B}T\ll E_{g}),n\approx(n_{0}N_{d})^{\frac{2}{1}}e^{\frac{-E_{d}}{k_{B}T}}(n_{0}=2(\frac{m_{e}k_{B}T}{2\pi\hbar^{2}})^{\frac{3}{2}})$.霍尔系数$R_{H}=\frac{-1}{nec}$\textbf{(11)双载流(i)霍尔}迁移率$\mu_{h}=\frac{e\tau_{h}}{m_{h}c},\mu_{e}=\frac{e\tau_{e}}{m_{e}c}.\vec{E}=E_{x}\hat{x},\vec{B}=B_{z}\hat{z}.(E_{y})_{h}=\frac{e\tau_{h}B_{z}E_{x}}{m_{h}c}=\frac{\mu_{h}B_{z}E_{x}}{c},(E_{y})_{h}=\frac{-e\tau_{e}B_{z}E_{x}}{m_{e}c}=\frac{-\mu_{e}B_{z}E_{x}}{c}.\hat{y}:(j_{y})_{h}=\sigma_{h}(E_{y})_{h}=-pe\mu_{h}\frac{\mu_{h}B_{z}E_{x}}{c},(j_{y})_{e}=\sigma_{e}(E_{y})_{e}=ne\mu_{e}\frac{\mu_{e}B_{z}E_{x}}{c}.j_{y}=(j_{y})_{h}+(j_{y})_{e}=\frac{e}{c}(n\mu_{e}^{2}-p\mu_{h}^{2})B_{z}E_{x}.$霍尔电场$\sigma E_{y}+j_{y}=0,E_{y}=\frac{(p\mu_{h}^{2}-n\mu_{e}^{2})B_{z}E_{x}}{c(p\mu_{h}+n\mu_{e})}.j_{x}=(pe\mu_{h}+ne\mu_{e})E_{x},E_{y}=\frac{(p-nb^{2})B_{z}j_{x}}{(p+nb)^{2}ec}(b=\frac{\mu_{e}}{\mu_{h}}).R_{H}=\frac{E_{y}}{j_{x}B_{z}}$.\textbf{(ii)磁致}$\mathbf{R}.$强$\vec{B}(\omega_{c}\tau\gg 1).m_{e}(\frac{\mathrm{d}}{\mathrm{d}t}+\frac{1}{\tau_{e}})v_{e}=-e(\vec{E}+\frac{\vec{v}_{e}}{c}\times\vec{B}),m_{h}(\frac{\mathrm{d}}{\mathrm{d}t}+\frac{1}{\tau_{h}})v_{h}=e(\vec{E}+\frac{\vec{v}_{h}}{c}\times\vec{B}).\vec{B}=B\hat{z}.$稳定:$\frac{\mathrm{d}v_{e}}{\mathrm{d}t}=\frac{\mathrm{d}v_{h}}{\mathrm{d}t}=0.v_{ex}=-\frac{e\tau_{e}E_{x}}{m_{e}}-\frac{e\tau_{e}Bv_{ey}}{cm_{e}}=-\frac{e\tau_{e}E_{x}}{m_{e}}-\omega_{e}\tau_{e}v_{ey};v_{ey}=-\frac{e\tau_{e}E_{y}}{m_{e}}+\omega_{e}\tau_{e}v_{ex};v_{ez}=\frac{-e\tau_{e}E_{x}}{m_{e}}(\omega_{e}=\frac{eB}{m_{e}c}).$解$v_{ex}=\frac{-e\tau_{e}(B_{x}-\omega_{e}\tau_{e}B_{y})}{m_{e}[1+(\omega_{e}\tau_{e})^{2}]},v_{ey}=\frac{-e\tau_{e}(E_{y}+\omega_{e}\tau_{e}E_{x})}{m_{e}[1+(\omega_{e}\tau_{e})^{2}]},v_{e}=\frac{-e\tau_{e}E_{x}}{m_{e}}$.空穴:变e,换h.设$a=\frac{e\tau_{e}}{m_{e}[1+(\omega_{e}\tau_{e})^{2}]},b=\frac{e\tau_{h}}{m_{h}[1+(\omega_{h}\tau_{h})^{2}]}.j=e(pv_{h}-nv_{e}),j_{x}=e[(pb+na)E_{x}+(pb\omega_{h}\tau_{h}-na\omega_{e}\tau_{e})E_{y}];j_{y}=e[(na\omega_{e}\tau_{e}-pb\omega_{h}\tau_{h})E_{x}+(pb+na)E_{y}];j_{z}=e^{2}[\frac{\tau_{h}p}{m_{h}}+\frac{\tau_{e}n}{m_{E}}]E_{z}.j_{y}=\sigma_{yx}E_{x}+\sigma_{yy}E_{y}+\sigma_{yz}E_{z},\sigma_{yx}=e(na\omega_{e}\tau_{e}-pb\omega_{h}\tau_{h})=e[\frac{ne\tau_{e}\omega_{e}\tau_{e}}{m_{e}[1+(\omega_{e}\tau_{e})^{2}]}-\frac{pe\tau_{h}\omega_{h}\tau_{h}}{m_{h}[1+(\omega_{h}\tau_{h})^{2}]}]\approx e[\frac{ne}{m_{e}\omega_{e}}-\frac{pe}{m_{h}\omega_{k}}]\approx\frac{ec(n-p)}{B}.$霍尔电场:$j_y=0,E_{y}=\frac{(pb\omega_{h}\tau_{h}-na\omega_{e}\tau_{e})E_{x}}{pb+na},Q_{i}=\omega_{i}\tau_{e}\ll 1,E_{y}\approx(\frac{pe\tau_{h}}{m_{h}Q_{h}^{2}}+\frac{ne\tau_{e}}{m_{e}Q_{e}^{2}})^{-1}(\frac{pe\tau_{h}}{m_{h}Q_{h}}-\frac{ne\tau_{e}}{m_{e}Q_{e}})E_{x}=-(n-p)(\frac{p}{Q_{h}}+\frac{n}{Q_{e}})^{-1}E_{x}.$有效$\sigma.j_{x}=\sigma_{\text{eff}}E_{x}=e[(pb+na)E_{x}+(pbQ_{h}-naQ_{e})E_{y}]\approx e[(\frac{pe\tau_{h}}{m_{h}Q_{h}^{2}}+\frac{ne\tau_{e}}{m_{e}Q_{e}^{2}})E_{x}+(\frac{pe\tau_{h}}{m_{h}Q_{h}}-\frac{ne\tau}{m_{e}Q_{e}})E_{y}],Q=\omega\tau=\frac{eB\tau}{mc},j_{x}\approx\frac{ec}{B}[(\frac{p}{Q_{h}}+\frac{n}{Q_{e}})E_{x}+(p-n)E_{y}]=\frac{ec}{B}[(\frac{p}{Q_{h}}+\frac{n}{Q_{e}})E_{x}+(p-n)^{2}(\frac{p}{Q_{h}}+\frac{n}{Q_{e}})^{-1}E_{x}]=\sigma_{\text{eff}}E_{x}.$\textbf{(12)椭色散与斜}$\mathbf{H}.\epsilon=\sum_{i}^{3}\frac{\hbar^{2}k_{i}^{2}}{2m_{i}},\vec{H}:\alpha,\beta,\gamma$(方向余弦).$\vec{v}(k)=\frac{\nabla_{k}E}{\hbar}=\sum_{i}\frac{\hbar k_{i}}{m_{i}}\hat{x}_{i},B=B(\alpha\hat{x}+\beta\hat{y}+\gamma\hat{z}).\frac{\hbar\mathrm{d}\vec{k}}{\mathrm{d}t}=-\hbar eB[(\frac{k_{1}\hat{x}}{m_{1}}+\frac{k_{2}\hat{y}}{m_{2}}+\frac{k_{3}\hat{z}}{m_{3}})\times(\alpha\hat{x}+\beta\hat{y}+\gamma\hat{z})].$方程组$\frac{\mathrm{d}k_{1}}{\mathrm{d}t}=-eB(\frac{k_{2}\alpha}{m_{2}}-\frac{k_{3}\beta}{m_{3}});\frac{\mathrm{d}\vec{k}_{2}}{\mathrm{d}t}=-eB(\frac{k_{3}\alpha}{m_{3}}-\frac{k_{1}\alpha}{m_{1}}),\frac{\mathrm{d}k_{3}}{\mathrm{d}t}-eB(\frac{k_{1}\beta}{m_{1}}-\frac{k_{2}\alpha}{m_{2}}).$试解$k_{i}=k_{i0}e^{i\omega t},$系数行列式:$\textbf{Det}[{i\omega,\frac{eB\alpha}{m_{2}},\frac{-eB\beta}{m_{3}}},{\frac{-eB\alpha}{m_{1}},i\omega,\frac{eB\alpha}{m_{3}}},{\frac{eB\beta}{m_{1}},\frac{-eB\alpha}{m_{2}},i\omega}]=0,\omega=eB\sqrt{\frac{\alpha^{2}}{m_{2}m_{3}}+\frac{\gamma^{2}}{m_{1}m_{2}}+\frac{\beta^{2}}{m_{1}m_{3}}}=\frac{Be}{m^{*}}$\textbf{(13)圆筒F面}$S=\frac{\pi r^{2}}{\cos{\alpha}},\Delta(\frac{1}{B})=\frac{2\pi e}{\hbar S}=\frac{2e\cos{\alpha}}{\hbar r^{2}}.$回旋质量:$\vec{B}=B\sin{\alpha}\hat{x}+B\cos{\alpha}\hat{z}.E=\frac{\hbar^{2}}{2m}(k_{x}^{2}+k_{y}^{2}).\vec{v}_{k}=\frac{\nabla E}{\hbar}=\frac{\hbar}{m}(k_{x}\hat{x}+k_{y}\hat{y}).$方程组$\frac{\mathrm{d}k_{x}}{\mathrm{d}t}=-\frac{eB\cos{\alpha}k_{y}}{m};\frac{\mathrm{d}k_{y}}{\mathrm{d}t}=\frac{eB\cos{\alpha}k_{x}}{m};\frac{\mathrm{d}k_{z}}{\mathrm{d}t}=\frac{eBk_{0}\sin{\alpha}}{m},k_{y}(t)=k_{y}(0)e^{\frac{ieB\cos{\alpha}t}{m}}.$\textbf{(14)非抛(ii)}色散:$\frac{\hbar^{2}|\vec{k}|^{2}}{2m}=E(1+\alpha E).\frac{\hbar^{2}|k|}{m}=\nabla_{k}E(1+2\alpha E),\vec{v}=\frac{\nabla_{k}E}{\hbar},\hbar\frac{\mathrm{d}k}{\mathrm{d}t}=-e\vec{v}\times\vec{B}=\frac{-e\hbar lB}{m(1+2\alpha E)},T=\frac{\oint\mathrm{d}k}{|\frac{\mathrm{d}k}{\mathrm{d}t}|}=\frac{2\pi k}{\frac{eBk}{m(1+2\alpha E)}}=\frac{2\pi m(1+2\alpha E)}{eB},\omega=\frac{eB}{m(1+2\alpha E)}, m_{c}^{*}=m(1+2\alpha E).$\textbf{(15)铁磁态(Jellium)}完全极化:$V'=2V,k_{F}'=2^{\frac{1}{3}}k_{F},\frac{3\epsilon_{F}}{5}\propto k_{F}^{2},\frac{3e^2k_{F}}{\pi}\propto k_{F},(2^{\frac{2}{3}}-1)\frac{2.21}{(\frac{r_{s}}{a_{0}})^{2}}-(2^{\frac{1}{3}}-1)\frac{0.916}{(\frac{r_{s}}{a_{0}})}<0,\frac{r_{s}}{a_{0}}>5.45$\textbf{(16)Jellium发散}$\phi_{i}=\frac{e^{ik_{i}r}}{\sqrt{\Omega}},E_{2}=\frac{-e^{2}}{2}\sum_{ij}\langle\phi_{i}\phi_{j}|\frac{1}{|\vec{r}-\vec{r}'|}|\phi_{i}\phi_{j}\rangle=\frac{-e^{2}N^{2}}{2\Omega^{2}}\int\frac{e^{-ik_{i}r}e^{-ik_{j}r'}e^{ik_{i}r}e^{ik_{j}r'}}{|\vec{r}-\vec{r}'|}\mathrm{d}\vec{r}\mathrm{d}\vec{r}'=\frac{-e^{2}N^{2}}{2\Omega^{2}}\int\frac{\mathrm{d}\vec{r}\mathrm{d}\vec{r}'}{|\vec{r}-\vec{r}'|}.$Avg.$\frac{E_{2}}{N}=-\frac{e^{2}N}{2\Omega^{2}}\int\frac{\mathrm{d}\vec{r}\mathrm{d}\vec{r}'}{|\vec{r}-\vec{r}'|}\propto\frac{N}{\Omega^{2}}\int\frac{\mathrm{d}(\vec{r}-\vec{r}')\mathrm{d}(\vec{r}+\vec{r})}{|\vec{r}-\vec{r}'|}\propto N^{\frac{3}{2}}.$

\end{document}