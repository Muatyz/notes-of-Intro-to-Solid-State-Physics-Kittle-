\documentclass[UTF8,a4paper,3pt,twocolumn]{ctexart}
\usepackage[left=0cm, right=2cm, top=0.2cm, bottom=0.2cm]{geometry}
%页边距
\CTEXsetup[format={\Large\bfseries}]{section} %设置章标题居左
%%%%%%%%%%%%%%%%%%%%%%%
% -- text font --
% compile using Xelatex
%%%%%%%%%%%%%%%%%%%%%%%
% -- 中文字体 --
%\setmainfont{Microsoft YaHei}  % 微软雅黑
%\setmainfont{YouYuan}  % 幼圆    
%\setmainfont{NSimSun}  % 新宋体
%\setmainfont{KaiTi}    % 楷体
%\setmainfont{SimSun}   % 宋体
%\setmainfont{SimHei}   % 黑体
% -- 英文字体 --
%\usepackage{times}
%\usepackage{mathpazo}
%\usepackage{fourier}
%\usepackage{charter}

%\usepackage{helvet}

\usepackage{amsmath, amsfonts, amssymb} % math equations, symbols
\usepackage[english]{babel}
\usepackage{color}	% color content
\usepackage{graphicx}	% import figures
\usepackage{url}	% hyperlinks
\usepackage{bm} 	% bold type for equations
\usepackage{multirow}
\usepackage{booktabs}
\usepackage{epstopdf}
\usepackage{epsfig}
\usepackage{algorithm}
\usepackage{algorithmic}
\usepackage{listings}
\usepackage{xcolor}
\usepackage{booktabs}
\usepackage{zhnumber}
\usepackage{longtable}
\usepackage{subfigure}
\usepackage{float}
\usepackage{caption}
\usepackage{subfigure}
\renewcommand\thesection{\zhnum{section}}
\renewcommand\thesubsection{\arabic{section}}
\renewcommand{\algorithmicrequire}{ \textbf{Input:}}
% use Input in the format of Algorithm  
\renewcommand{\algorithmicensure}{ \textbf{Initialize:}}
% use Initialize in the format of Algorithm  
\renewcommand{\algorithmicreturn}{ \textbf{Output:}}
% use Output in the format of Algorithm
%%简化编写公式难度
\usepackage{braket} %量子算符宏包
\usepackage{cancel}%表示项的消去 

%%%%%%%%%%%%%%%%%%
\usepackage{listings}
\usepackage{color}
\definecolor{dkgreen}{rgb}{0,0.6,0}
\definecolor{gray}{rgb}{0.5,0.5,0.5}
\definecolor{mauve}{rgb}{0.58,0,0.82}
\lstset{frame=tb,
  language=Python,
  aboveskip=3mm,
  belowskip=3mm,
  showstringspaces=false,
  columns=flexible,
  basicstyle={\small\ttfamily},
  numbers=left,%设置行号位置none不显示行号
  %numberstyle=\tiny\courier, %设置行号大小
  numberstyle=\tiny\color{gray},
  keywordstyle=\color{blue},
  commentstyle=\color{dkgreen},
  stringstyle=\color{mauve},
  breaklines=true,
  breakatwhitespace=true,
  escapeinside=``,%逃逸字符(1左面的键),用于显示中文例如在代码中`中文...`
  tabsize=4,
  extendedchars=false %解决代码跨页时,章节标题,页眉等汉字不显示的问题
}

%%%%%%%%%%%%%%%%%%%%%%%%%%%%
\usepackage{fancyhdr} %设置页眉、页脚

\lhead{}
\chead{}
%\rhead{\includegraphics[width=1.2cm]{fig/ZJU_BLUE.eps}}
\lfoot{}
\cfoot{}
\rfoot{}

%%%%%%%%%%%%%%%%%%%%%%%
%  设置水印
%%%%%%%%%%%%%%%%%%%%%%%
%\usepackage{draftwatermark}         % 所有页加水印
%\usepackage[firstpage]{draftwatermark} % 只有第一页加水印
% \SetWatermarkText{Water-Mark}           % 设置水印内容
% \SetWatermarkText{\includegraphics{fig/ZJDX-WaterMark.eps}}         % 设置水印logo
% \SetWatermarkLightness{0.9}             % 设置水印透明度 0-1
% \SetWatermarkScale{1}                   % 设置水印大小 0-1    

\usepackage{hyperref} %bookmarks
\hypersetup{colorlinks, bookmarks, unicode} %unicode

\begin{document}
\begin{itemize}
  \item \textbf{积分公式}.$\int_{-\infty}^{\infty}exp[ix^2]dx=\sqrt{\pi}exp[i\pi/4]$(Fresnel积分公式);
  $\int_{-\infty}^{\infty}dxexp[-\alpha x^2+\beta x]=\sqrt{\frac{\pi}{\alpha}}exp[\frac{\beta^2}{4\alpha}]$,
  $\int_{0}^{+\infty}x^{n}exp[-ax^{2}]dx=\frac{\Gamma(\frac{n+1}{2})}{2a^{\frac{n+1}{2}}}$,
  $\int_{-\infty}^{+\infty}xexp[-\frac{1}{2}ax^2+bx]dx=\frac{b}{a}\sqrt{\frac{2\pi}{a}}exp[b^2/(2a)]$,
  $\int_{-\infty}^{+\infty}x^{2}exp[-\frac{1}{2}ax^2+bx]dx=\frac{1}{a}(1+\frac{b^{2}}{a})\sqrt{\frac{2\pi}{a}}exp[b^{2}/(2a)]$;
  $\int_{-\infty}^{+\infty}x^{2n}exp[-\frac{1}{2}ax^{2}]dx=\frac{(2n-1)!!}{a^{n}}\sqrt{\frac{2\pi}{a}}$(广义Guass积分式);
  $\int_{0}^{+\infty}x^{2n+1}exp[-ax^{2}]dx=\frac{n!}{2a^{n+1}}$;
  $(\frac{1}{\sqrt{2\pi\hbar}})^{3}\iiint exp[-\frac{i}{\hbar}\vec{p}'\cdot\vec{r}](p_{z}\frac{\partial}{\partial p_{y}}-p_{y}\frac{\partial}{\partial p_{z}})exp[\frac{i}{\hbar}\vec{p}\cdot\vec{r}]d\tau=
  (p_{z}\frac{\partial}{\partial p_{y}}-p_{y}\frac{\partial}{\partial p_{z}})(\frac{1}{\sqrt{2\pi\hbar}})^{3}\iiint exp[\frac{i}{\hbar}(\vec{p}-\vec{p}')\cdot\vec{r}]d\tau
  =(p_{z}\frac{\partial}{\partial p_{y}}-p_{y}\frac{\partial}{\partial p_{z}})\delta(\vec{p}-\vec{p}')$

  \item \textbf{晶格}  (1)三斜(1;$a_1\neq a_2\neq a_3;\alpha\neq\beta\neq\gamma$);单斜(2;$a_1\neq a_2\neq a_3;\alpha=\gamma=\pi/2\neq\beta$);
  正交(4;$a_1\neq a_2\neq a_3;\alpha=\beta=\gamma=\pi/2$);四角(2,$a_1=a_2\neq a_3;\alpha=\beta=\gamma=\pi/2$);
  立方(3;$a_1=a_2=a_3;\alpha=\beta=\gamma=\pi/2$);三角(1,$a_1=a_2=a_3;\alpha=\beta=\gamma\neq\pi/2$);
  六角(1;$a_1=a_2\neq a_3;\alpha=\beta=\pi/2,\gamma=2\pi/3$)(2)sc(简单立方);bcc(体心立方);fcc(面心立方);hcp(六角密堆积)
  (3)常见结构:NaCl($Cl^{-}$面心\&角+$Na^{+}$边中\&体心);CsCl($Cs^{+}$体心+$Cl^{-}$角);
  金刚石结构(fcc+$000\&\frac{1}{4}\frac{1}{4}\frac{1}{4}$);ZnS结构(Zn$000,0\frac{1}{2}\frac{1}{2},\frac{1}{2}0\frac{1}{2},\frac{1}{2}\frac{1}{2}0$;
  S$\frac{1}{4}\frac{1}{4}\frac{1}{4},\frac{1}{4}\frac{3}{4}\frac{3}{4},\frac{3}{4}\frac{1}{4}\frac{3}{4},,\frac{3}{4}\frac{3}{4}\frac{1}{4}$)

  \item \textbf{两种指标}设晶面截距为$a_1,a_2,a_3$(1)$(a_1^{-1},a_{2}^{-1},a_2^{-1})$;
  (2)$[a_1,a_2,a_3]$.
  上划线表示负号$[u\overline{v}w]$
  
  \item \textbf{布拉格条件}$2d\sin{\theta}=n\lambda;\Delta\vec{k}=\vec{G};2\vec{k}\cdot\vec{G}=\vec{G}^2$;
  \item \textbf{劳厄条件}$\vec{a_1}\cdot\Delta\vec{k}=2\pi v_1;\vec{a_2}\cdot\Delta\vec{k}=2\pi v_2;\vec{a_3}\cdot\Delta\vec{k}=2\pi v_3$;
  \item \textbf{倒格子初基平移矢量}$\vec{b_1}=2\pi\frac{\vec{a_2}\times\vec{a_3}}{\vec{a_1}\cdot\vec{a_2}\times\vec{a_3}},
  \vec{b_2}=2\pi\frac{\vec{a_3}\times\vec{a_1}}{\vec{a_1}\cdot\vec{a_2}\times\vec{a_3}},
  \vec{b_3}=2\pi\frac{\vec{a_1}\times\vec{a_2}}{\vec{a_1}\cdot\vec{a_2}\times\vec{a_3}}$

  \item \textbf{倒格矢}$\vec{G}=v_1\vec{b_1}+v_2\vec{b_2}+v_3\vec{b_3},v_{i}为\mathcal{Z}$
  \item \textbf{几何结构因子}前提:方向为$\vec{k'}=\vec{k}+\Delta\vec{k}=\vec{k}+\vec{G}$,
  $S_{G}=\sum_{j} f_{j}e^{-i\vec{r_{j}}\cdot\vec{G}}=\sum_{j}e^{-i2\pi(x_j v_1+y_j y_2+z_j v_3)}$,
  其中$f_{j}=\int dVn_{j}(\vec{r})e^{-i\vec{G}\cdot\vec{r}}$
  \item \textbf{第一布里渊区}倒格子的维格纳-塞茨原胞(1)sc$-\rangle$sc($2\pi/a$);bcc$-\rangle$棱形十二面体($2\pi/a\sqrt{2}$);
  fcc$-\rangle$截角八面体(八面体的每个角都被切下,使得相邻三个面的正方形的边能围成正六边形)

  \item \textbf{声子-振动}
  (1)单原子:$u_{s\pm 1}=ue^{isKa}exp^{\pm iKa}$色散关系:$w^2=(2C/M)(1-\cos{Ka})$;
  $w^2=(4C/M)\sin^2{\frac{1}{2}Ka}$;群速:$v_{g}=\frac{dw}{dK}=\sqrt{\frac{Ca^2}{M}}\cos{\frac{1}{2}Ka}$;
  长波极限($Ka<<1$):$w^2=(C/M)K^{2}a^{2}$
  (2)双原子:原胞p个原子,3个声学支,3p-3个光学支.$M_1\frac{d^2 u_s}{dt^2}=C(v_s+v_{s-1}-2u_s);M_2\frac{d^2 v_s}{dt^2}=C(u_{s+1}+u_{s}-2v_s)$.
  $u_s=ue^{isKa}e^{-iwt},v_s=ve^{isKa}e^{-iwt}$,行列式系数为0:$M_1 M_2 w^4-2C(M_1+M_2)w^2+2C^2(1-\cos{Ka})=0$;长波极限:
  光学支$w^2=2C(\frac{1}{M_1}+\frac{1}{M_2})$,声学支$w^2=\frac{C}{2(M_1+M_2)}K^2 a^2$;光学支下原子反向震动即质心固定,由光的电场来激发.
  (3)波矢选择定则:波矢$\vec{k}$非弹性散射到$\vec{k'}$,同时产生/吸收波矢为$\vec{K}$的声子,那么$\vec{k}=\vec{k'}\pm\vec{K}+\vec{G}$,$\vec{G}$是倒格矢;
  (4)声子能量:$\epsilon=(n+\frac{1}{2})\hbar\omega$;动能守恒:$\frac{\hbar^2 k^2}{2M_n}=\frac{\hbar^2 k'^2}{2M_n}\pm\hbar\omega$

  \item \textbf{热学基础}
  (0)定容热容$C_V=(\frac{\partial U}{\partial T})_V$,声子温度为$\tau=k_{B}T$,晶格内能$U_{lat}\sum_{K}\sum_{p}\langle n_{K,p}\rangle\hbar\omega_{K,p}$
  (1)普朗克分布$\langle n\rangle=\frac{1}{e^{\frac{\hbar\omega}{\tau}}-1}$
  (2)$U=\sum_{K}\sum_{p}\frac{\hbar\omega_{K,p}}{e^{\frac{\hbar\omega_{K,p}}{\tau}}-1}=\sum_{p}\int d\omega D_{p}(\omega)\frac{\hbar\omega}{e^{\frac{\hbar\omega}{\tau}}-1}$,
  $C_{lat}=k_{B}\sum_{p}\int d\omega D_{p}(\omega)\frac{x^2 e^x}{(e^x -1)^2}$($x=\hbar\omega/\tau=\hbar\omega/k_{B}T$),$D(\omega)$即为态密度
  (3)一维$D(\omega)$:$L=Na$,每个间隔$\Delta K=\frac{\pi}{L}$内一个模式,每个$K$三个偏振态(两个横向一个纵向)$D(\omega)d\omega=\frac{L}{\pi}\frac{dK}{d\omega}d\omega
  =\frac{L}{\pi}\frac{d\omega}{d\omega/dK}$(已知色散关系$\omega(K)$)
  (4)三维$D(\omega)$:$\forall i, K_{i}=\pm\frac{2n\pi}{L}$,$\vec{K}$的每单位体积内的模式数为$(\frac{L}{2\pi})^3=\frac{V}{8\pi^3}$,
  每种偏振模式总数$N=(\frac{L}{2\pi})^3(\frac{4\pi K^3}{3})$,态密度$D(\omega)=\frac{dN}{d\omega}=(\frac{VK^2}{2\pi^2})(\frac{dK}{d\omega})$
  
  \item \textbf{德拜模型}
  (1)假设(每种偏振声速恒定,$\omega=vK$)$D(\omega)=\frac{V\omega^2}{2\pi^{2}v^3}$,
  截止频率$\omega_{D}^3=6\pi^{2}v^{3}N/V$,截止波矢$K_{D}=\omega_{D}/v=(6\pi^{2}\frac{N}{V})^{\frac{1}{3}}$,
  单偏振态内能$U_i=\int d\omega D(\omega)\langle n(\omega)\rangle\hbar\omega
  =\int_{0}^{\omega_{D}}d\omega(\frac{V\omega^{2}}{2\pi^{2}v^{3}})(\frac{\hbar\omega}{e^{\frac{\hbar\omega}{\tau}}-1})$,
  总内能$U=3U_i=\frac{3V\hbar}{2\pi^2 v^3}\int_{0}^{\omega_{D}}d\omega\frac{\omega^3}{e^{\frac{\hbar\omega}{\tau}}-1}$
  $=\frac{3Vk_{B}^{4}T^{4}}{2\pi^2 v^3 \hbar^3}\int_{0}^{x_{D}}dx\frac{x^3}{e^x - 1}$(其中$x=\hbar\omega/\tau$,$x_{D}=\hbar\omega_{D}/\tau=\theta/T$),
  德拜温度$\theta=\frac{\hbar v}{k_{B}}(\frac{6\pi^2N}{V})^{\frac{1}{3}}$,$U=9Nk_{B}T(\frac{T}{\theta})^3\int_{0}^{x_D}dx\frac{x^3}{e^x -1}$
  (2)德拜模型低温极限($T^3$率)($\int_{0}^{\infty}dx\frac{x^3}{e^x -1}=\frac{\pi^4}{15}$):$U\approxeq3\pi^2 Nk_{B}T^4/5\theta^3$,热容$C_{V}\approxeq\frac{12\pi^4}{5}Nk_{B}(\frac{T}{\theta})^3\approxeq 234Nk_{B}(\frac{T}{\theta})^3$
  
  \item \textbf{爱因斯坦模型}
  爱因斯坦模型($D(\omega)=N\delta(\omega-\omega_{0})$):一维内能$U=n\langle n\rangle\hbar\omega=\frac{N\hbar\omega}{e^{\frac{\hbar\omega}{\tau}}-1}$,
  一维比热$C_{V}=\frac{\partial U}{\partial T}_{V}=Nk_{B}(\frac{\hbar\omega}{\tau})^2\frac{e^{\hbar\omega/\tau}}{(e^{\hbar\omega/\tau}-1)^2}$.三维乘系数3.
  
  \item \textbf{声子热学}
  (1)\textbf{态密度}$D(\omega)$一般形式:$D(\omega)=\frac{V}{(2\pi)^3}\int_{\text{K空间中}\omega\text{恒定的曲面}}\frac{dS_{\omega}}{v_{g}}$
  (2)非谐作用($U(x)=cx^2-gx^3-fx^4$):$\langle x\rangle=\frac{\int_{-\infty}^{+\infty}dx xe^{-\beta U(x)}}{\int_{-\infty}^{+\infty}dx e^{-\beta U(x)}}$($\beta=\frac{1}{k_{B}T}$),
  $\int dx xe^{-\beta U}\approxeq(\frac{3\pi^{\frac{1}{2}}}{4})(\frac{g}{c^{\frac{5}{2}}})\beta^{-\frac{3}{2}},\int dxe^{-\beta U}\approxeq(\frac{\pi}{\beta c})^{\frac{1}{2}}$,
  $\langle x\rangle=\frac{3g}{4c^2}k_{B}T$
  (3)热导.一维下热流量$j_{U}=-K\frac{dT}{dx}$,热导率$K=\frac{1}{3}Cvl$($C$:单位体积比热;$v$:粒子平均速度;$l$:平均自由程).
  (4)过程.$\vec{K}_{1}+\vec{K}_{2}=\vec{K}_{3}+\vec{G}$.正常过程($N$):$\vec{G}=0$;倒逆过程($U$):$\vec{G}\neq 0$

  \item \textbf{自由电子}
  (0)一维无限深井:$\mathcal{H}\psi_{n}=-\frac{\hbar^2}{2m}\frac{d^2\psi_{n}}{dx^2}=\epsilon_{n}\psi_{n};\epsilon_{n}=\frac{\hbar^2}{2m}(\frac{n\pi}{L})^2$
  (1)费米能$\epsilon_{F}$:N电子系统基态下的最高能级;e.g.一维无限深井+泡利原理:$2n_F=N,n=n_F,\epsilon_F=\frac{\hbar^2}{2m}(\frac{N\pi}{2L})^2$;
  (2)温度变量.$f(\epsilon,T,\mu)=\frac{1}{e^{[\epsilon-\mu(T)]/k_{B}T}+1}$($T=0$时$\mu=\epsilon_F$).取高温极限时成为玻尔兹曼分布或者麦氏分布.
  (3)三维:$-\frac{\hbar^2}{2m}\nabla^2\psi_{k}(\vec{r})=\epsilon_{\vec{k}}\psi_{k}(\vec{r}),\psi_{\vec{k}}(\vec{r})=e^{i\vec{k}\cdot\vec{r}},(\forall i,k_i=\frac{2n\pi}{L})$
  $\epsilon_{\vec{k}}=\frac{\hbar^2}{2m}(k_x^2+k_y^2+k_z^2)$.$\hat{p}\psi_{\vec{k}}(\vec{r})=\hbar\vec{k}\psi_{\vec{k}}(\vec{r}),\vec{v}=\frac{\hbar\vec{k}}{m}$.
  费米波矢$k_F$,费米能$\epsilon_F=\frac{\hbar^2}{2m}k_{F}^2$.$k$空间的每个体积元$(\frac{2\pi}{L})^3$存在一个波矢$(k_x,k_y,k_z)$.费米球+泡利定理:$2\cdot\frac{4\pi k_F^2/3}{(2\pi/L)^3}=N$.
  费米波矢$k_F=(\frac{3\pi^2N}{V})^{\frac{1}{3}}$,费米能$\epsilon_{F}=\frac{\hbar^2}{2m}(\frac{3\pi^2N}{V})^{\frac{2}{3}}$,费米速度$v_F=(\frac{\hbar k_F}{m})=\frac{\hbar}{m}(\frac{3\pi^2N}{V})^{\frac{1}{3}}$.
  费米温度$T_F=\epsilon_F/k_B$.$N(U\leq\epsilon)=\frac{V}{3\pi^2}(\frac{2m\epsilon}{\hbar^2})^{\frac{3}{2}},D(\epsilon)=\frac{dN}{d\epsilon}=\frac{V}{2\pi^2}(\frac{2m}{\hbar^2})^{\frac{3}{2}}\epsilon^{\frac{1}{2}}=\frac{3N}{2\epsilon}$
  (4)比热容.总电子内能$U_{e}\approx\frac{NT}{T_F}k_B T$,电子比热$C_{e}=\frac{\partial U}{\partial T}\approx Nk_B \frac{T}{T_F}$.
  低温极限($k_B T\ll \epsilon$):$\Delta U=\int_0^{\infty}d\epsilon\epsilon D(\epsilon)f(\epsilon)-\int_0^{\epsilon_F}d\epsilon\epsilon D(\epsilon)$
  $=\int_{\epsilon_F}^{\infty}d\epsilon(\epsilon-\epsilon_F)f(\epsilon)D\epsilon + \int_0^{\epsilon_F}d\epsilon(\epsilon_F-\epsilon)[1-f(\epsilon)]D(\epsilon)$.
  电子热容$C_e=\frac{dU}{dT}=\int_{0}^{\infty}d\epsilon(\epsilon-\epsilon_F)\frac{df}{dT}D(\epsilon)\approx D(\epsilon_F)\int_0^{\infty}d\epsilon(\epsilon-\epsilon_F)\frac{df}{dT}$
  低温极限($\tau=k_B T,x=\frac{\epsilon-\epsilon_F}{\tau}$)$\int_{-\infty}^{+\infty}dxx^2\frac{e^x}{(e^x +1)^2}=\frac{\pi^2}{3},C_e=\frac{1}{3}\pi^2D(\epsilon_F)k_B^2T$
  ($D(\epsilon_F)=\frac{3N}{2\epsilon_F}$),$C_e=\frac{1}{2}\pi^2Nk_BT/T_F$.
  (5)金属比热.$\frac{C}{T}=\gamma+AT^2$($\gamma$索末菲常量).
  (6)电导率.$\vec{F}=-e(\vec{E}+\frac{1}{c}\vec{v}\times\vec{B})$.若$\vec{F}=-e\vec{E},\delta\vec{k}=-e\vec{E}t/\hbar,\vec{v}=\delta\vec{k}/m=-e\vec{E}\tau/m$.
  电流密度$\vec{j}=nq\vec{v}=ne^2\tau\vec{E}/m$,($\vec{j}=\sigma\vec{E}$)$\sigma=\frac{ne^2\tau}{m},\rho=\sigma^{-1}$.
  (7)磁场下运动.(CGS制)$\hbar(\frac{d}{dt}+\frac{1}{\tau})\delta\vec{k}=\vec{F}=-e(\vec{E}+\vec{v}\times\vec{E})$.若$\vec{B}\parallel\hat{z}$,
  $\{v_x=-\frac{e\tau}{m}E_x-\omega_c\tau v_y,v_y=-\frac{e\tau}{m}E_y+\omega_c\tau v_x,v_z=-\frac{e\tau}{m}E_z\}$,回旋频率$\omega_c=\frac{eB}{mc}$
  (8)霍尔效应.霍尔系数$R_H=\frac{E_y}{j_x B}=-\frac{1}{nec}$(CGS).
  (9)金属热导率.$K_e=\frac{1}{3}Cvl=\frac{\pi^2}{3}\frac{nk_B^2T}{mv_F^2}v_Fl=\frac{\pi^2 n k_B^2 T\tau}{3m}$

  \item \textbf{近自由电子模型}
  (0)一维晶体.布拉格衍射条件$(\vec{k}+\vec{G})^2=\vec{k}^2\rightarrow k=\pm\frac{1}{2}G=\pm\frac{n\pi}{a}$(倒格矢$G=\frac{2\pi n}{a}$)
  (1)驻波.与时间无关.$\psi(+)=e^{i\pi x/a}+e^{-i\pi x/a}=2\cos{\pi x/a},\psi(-)=e^{i\pi x/a}-e^{-i\pi x/a}=2i\sin{\pi x/a}$.
  $\rho(+)=|\psi(+)|^2\propto\cos^2{\pi x/a},\rho(-)=|\psi(-)|^2\propto\sin^2{\pi x/a}$.
  大小关系:$\langle\psi(-)|U|psi(-)\rangle<\langle e^{\mp i\pi x/a}|U|e^{\pm i\pi x/a}\rangle<\langle\psi(+)|U|psi(+)\rangle$.
  若一维$\psi(x)=\sqrt{2}\cos{\pi x/a},\sqrt{2}\sin{\pi x/a}$,电子势能$U(x)=U\cos{2\pi x/a}$,则能隙$E_g=U(+)-U(-)=\int_{0}^{1}dx U(x)[|\psi(+)|^2-|\psi(-)|^2]=U$.
  (2)布洛赫函数.若势周期,则$\psi_{\vec{k}}(\vec{r})=u_{\vec{k}}(\vec{r})e^{i\vec{k}\cdot\vec{r}}$(其中$u_{\vec{k}}(\vec{r})=u_{\vec{k}}(\vec{r}+\vec{T})$).
  若非简并,$\psi(x+a)=C\psi(x),C=e^{i2\pi s/N}->\psi(x)=u_{\vec{k}}(x)e^{i2\pi sx/N}$.
  (3)克勒尼希-彭尼模型Kölnig-Penney model(周期势阱).$-\frac{\hbar^2}{2m}\frac{d^2\psi}{dx^2}+U(x)\psi=\epsilon\psi$.
  $x\in(0,a):\psi=Ae^{iKx}+B^{-iKx},\epsilon=\frac{\hbar^2 K^2}{2m};x\in(-b,0):\psi=Ce^{Qx}+De^{-Qx},U_0-\epsilon=\frac{\hbar^2 Q^2}{2m}$.
  函数连续+导数连续,有四阶系数行列式为0:$[(Q^2-K^2)/2QK]\sinh{Qb}\sin{Ka}+\cosh{Qb}\cos{Ka}=\cos{k(a+b)}$.
  取极限$b=0,U_0=\infty(Q\gg K,Qb\ll 1)$,即为周期性$\delta$函数,$P=\frac{Q^2ba}{2}$结论化为$(P/Ka)\sin{Ka}+\cos{Ka}=\cos{ka}$.
  (4)周期势下的电子波函数.$U(x)=\sum_{G}U_{G}e^{iGx}$,若为实则$U(x)=\sum_{G>0}2U_G\cos{Gx}$.$\psi=\sum_{k}C(k)e^{ikx}$.
  波动方程$\sum_{k}\frac{\hbar^2}{2m}k^2C(k)e^{ikx}+\sum_{G}\sum_{k}U_G C(k)e^{i(k+G)x}=\epsilon\sum_{k}e^{ikx}$.
  中心方程$(\lambda_k-\epsilon)C(k)+\sum_{G}U_GC(k-G)=0$(其中$\lambda_k=\frac{\hbar^2 k^2}{2m}$)
  (5)求解行列式
  $$
  \left |\begin{array}{ccc}
   \lambda_{k-g}-\epsilon & U & 0  \\
   U & \lambda_{k}-\epsilon & U  \\
   0 & U & \lambda_{k+g}-\epsilon   \\
        \end{array}\right|
  $$
  每一个$k$每个$\epsilon$将在不同能带上.
  (6)中心方程求解K-P方程(周期$\delta$势函数).$U(x)=Aa\sum_{s}\delta(x-sa),U_G=\int_0^{1}dxU(x)cos(Gx)=A.$
  中心方程变为$(\lambda_k-\epsilon)C(k)+Af(k)=0$,其中$f(k)=\sum_{n}C(k-2\pi n/a)=f(k\pm 2\pi n/a)$.
  从而有$\frac{mAa^2}{2\hbar^2}(Ka)^{-1}\sin{Ka}+\cos{Ka}=\cos{ka}$.
  (7)布里渊区边界附近的近似解.
  $k^2=(\frac{1}{2}G)^2,(k-G)^2=(\frac{1}{2}G-G)^2\rightarrow k=\pm\frac{1}{2}G$.
  ($k=\frac{1}{2}G,\lambda=\hbar^2(\frac{1}{2}G)^2/2m$)$(\lambda-\epsilon)C(\pm\frac{1}{2}G)+UC(\mp\frac{1}{2}G)=0$.
  行列式$|^{\lambda-\epsilon,U}_{U,\lambda-\epsilon}|=0$,解得$\epsilon=\lambda\pm U,E_g=2U$.
  若在$\frac{1}{2}G$附近,则$(\lambda_k-\epsilon)C(k)+UC(k-G)=0,(\lambda_{k-G})C(k-G)+UC(k)=0(\lambda_k=\hbar^2k^2/2m)$,
  系数行列式$|^{\lambda_k-\epsilon,U}_{U,\lambda_{k-G}-\epsilon}|=0\rightarrow\epsilon=\frac{1}{2}(\lambda_{k-G}+\lambda_k)\pm[\frac{1}{4}(\lambda_{k-G}-\lambda_k)^2+U^2]^\frac{1}{2}$
  用小量$\widetilde{K}=k-\frac{1}{2}G$展开,有$\epsilon_{\widetilde{K}}\approx\frac{\hbar^2}{2m}(\frac{1}{4}G^2+\widetilde{K}^2)\pm U[1+2(\frac{\lambda}{U^2})(\frac{\hbar^2\widetilde{K}^2}{2m})]$.
  (8)轨道数目.N原胞一维晶体:$k=\pm\frac{2n\pi}{L}$.每个原胞对应一个k+泡利定理$\rightarrow$每个能带2N个轨道.

\end{itemize}

  % \begin{table}[htbp]
  %   \caption{布拉维格子及其倒格子的几何参数}
  %   \centering
  %   \begin{tabular}{c|c|c|c|c}
  %   \hline
  %   & 点群 & 简写 & 布拉维格子几何参数 & 倒格子几何参数 \\
  %   \hline
  %   \hline
  %   立方晶系 & m3m & fcc & a & $\frac{2\pi}{a}$ \\
  %   & & bcc & a & $\frac{2\pi}{a}\sqrt{\frac{2}{3}}$ \\
  %   \hline
  %   正交晶系 & mmm & & a, b, c & $\frac{2\pi}{a}$, $\frac{2\pi}{b}$, $\frac{2\pi}{c}$ \\
  %   \hline
  %   单斜晶系 & 2/m & & a, b, c, $\beta$ & $\frac{2\pi}{a}$, $\frac{2\pi}{b}$, $\frac{2\pi}{c}$, $\alpha^{},\beta^{},\gamma^{}$ \\
  %   \hline
  %   三斜晶系 & 1 & & a, b, c, $\alpha$, $\beta$, $\gamma$ & $\alpha^{},\beta^{} ,\gamma^{}$ \\
  %   \hline
  %   菱面体晶系 & m3m & & a, c & $\frac{2\pi}{a}$, $\frac{2\pi}{c}$ \\
  %   \hline
  %   正四面体晶系 & 23 & & a & $\frac{4\pi}{a}$ \\
  %   \hline
  %   正十二面体晶系 & m3 & & a & $\frac{4\pi}{a}$ \\
  %   \hline
  %   \end{tabular}
  %   \end{table}



\end{document}