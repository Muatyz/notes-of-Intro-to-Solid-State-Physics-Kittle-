\documentclass[UTF8,a4paper,1pt]{ctexart}
\usepackage[left=0cm, right=2cm, top=0.2cm, bottom=0.2cm]{geometry}
%页边距
\CTEXsetup[format={\Large\bfseries}]{section} %设置章标题居左
%%%%%%%%%%%%%%%%%%%%%%%
% -- text font --
% compile using Xelatex
%%%%%%%%%%%%%%%%%%%%%%%
% -- 中文字体 --
%\setmainfont{Microsoft YaHei}  % 微软雅黑
%\setmainfont{YouYuan}  % 幼圆    
%\setmainfont{NSimSun}  % 新宋体
%\setmainfont{KaiTi}    % 楷体
%\setmainfont{SimSun}   % 宋体
%\setmainfont{SimHei}   % 黑体
% -- 英文字体 --
%\usepackage{times}
%\usepackage{mathpazo}
%\usepackage{fourier}
%\usepackage{charter}

%\usepackage{helvet}

\usepackage{amsmath, amsfonts, amssymb} % math equations, symbols
\usepackage[english]{babel}
\usepackage{color}	% color content
\usepackage{graphicx}	% import figures
\usepackage{url}	% hyperlinks
\usepackage{bm} 	% bold type for equations
\usepackage{multirow}
\usepackage{booktabs}
\usepackage{epstopdf}
\usepackage{epsfig}
\usepackage{algorithm}
\usepackage{algorithmic}
\usepackage{listings}
\usepackage{xcolor}
\usepackage{booktabs}
\usepackage{zhnumber}
\usepackage{longtable}
\usepackage{subfigure}
\usepackage{float}
\usepackage{caption}
\usepackage{subfigure}
\renewcommand\thesection{\zhnum{section}}
\renewcommand\thesubsection{\arabic{section}}
\renewcommand{\algorithmicrequire}{ \textbf{Input:}}
% use Input in the format of Algorithm  
\renewcommand{\algorithmicensure}{ \textbf{Initialize:}}
% use Initialize in the format of Algorithm  
\renewcommand{\algorithmicreturn}{ \textbf{Output:}}
% use Output in the format of Algorithm
%%简化编写公式难度
\usepackage{braket} %量子算符宏包
\usepackage{cancel}%表示项的消去 

%%%%%%%%%%%%%%%%%%
\usepackage{listings}
\usepackage{color}
\definecolor{dkgreen}{rgb}{0,0.6,0}
\definecolor{gray}{rgb}{0.5,0.5,0.5}
\definecolor{mauve}{rgb}{0.58,0,0.82}
\lstset{frame=tb,
  language=Python,
  aboveskip=3mm,
  belowskip=3mm,
  showstringspaces=false,
  columns=flexible,
  basicstyle={\small\ttfamily},
  numbers=left,%设置行号位置none不显示行号
  %numberstyle=\tiny\courier, %设置行号大小
  numberstyle=\tiny\color{gray},
  keywordstyle=\color{blue},
  commentstyle=\color{dkgreen},
  stringstyle=\color{mauve},
  breaklines=true,
  breakatwhitespace=true,
  escapeinside=``,%逃逸字符(1左面的键),用于显示中文例如在代码中`中文...`
  tabsize=4,
  extendedchars=false %解决代码跨页时,章节标题,页眉等汉字不显示的问题
}

%%%%%%%%%%%%%%%%%%%%%%%%%%%%
\usepackage{fancyhdr} %设置页眉、页脚

\lhead{}
\chead{}
%\rhead{\includegraphics[width=1.2cm]{fig/ZJU_BLUE.eps}}
\lfoot{}
\cfoot{}
\rfoot{}

%%%%%%%%%%%%%%%%%%%%%%%
%  设置水印
%%%%%%%%%%%%%%%%%%%%%%%
%\usepackage{draftwatermark}         % 所有页加水印
%\usepackage[firstpage]{draftwatermark} % 只有第一页加水印
% \SetWatermarkText{Water-Mark}           % 设置水印内容
% \SetWatermarkText{\includegraphics{fig/ZJDX-WaterMark.eps}}         % 设置水印logo
% \SetWatermarkLightness{0.9}             % 设置水印透明度 0-1
% \SetWatermarkScale{1}                   % 设置水印大小 0-1    

\usepackage{hyperref} %bookmarks
\hypersetup{colorlinks, bookmarks, unicode} %unicode

\begin{document}
\begin{itemize}
  \item \textbf{积分公式}.$\int_{-\infty}^{\infty}exp[ix^2]dx=\sqrt{\pi}exp[i\pi/4]$(Fresnel积分公式);
  $\int_{-\infty}^{\infty}dxexp[-\alpha x^2+\beta x]=\sqrt{\frac{\pi}{\alpha}}exp[\frac{\beta^2}{4\alpha}]$,
  $\int_{0}^{+\infty}x^{n}exp[-ax^{2}]dx=\frac{\Gamma(\frac{n+1}{2})}{2a^{\frac{n+1}{2}}}$,
  $\int_{-\infty}^{+\infty}xexp[-\frac{1}{2}ax^2+bx]dx=\frac{b}{a}\sqrt{\frac{2\pi}{a}}exp[b^2/(2a)]$,
  $\int_{-\infty}^{+\infty}x^{2}exp[-\frac{1}{2}ax^2+bx]dx=\frac{1}{a}(1+\frac{b^{2}}{a})\sqrt{\frac{2\pi}{a}}exp[b^{2}/(2a)]$;
  $\int_{-\infty}^{+\infty}x^{2n}exp[-\frac{1}{2}ax^{2}]dx=\frac{(2n-1)!!}{a^{n}}\sqrt{\frac{2\pi}{a}}$(广义Guass积分式);
  $\int_{0}^{+\infty}x^{2n+1}exp[-ax^{2}]dx=\frac{n!}{2a^{n+1}}$;
  $(\frac{1}{\sqrt{2\pi\hbar}})^{3}\iiint exp[-\frac{i}{\hbar}\vec{p}'\cdot\vec{r}](p_{z}\frac{\partial}{\partial p_{y}}-p_{y}\frac{\partial}{\partial p_{z}})exp[\frac{i}{\hbar}\vec{p}\cdot\vec{r}]d\tau=
  (p_{z}\frac{\partial}{\partial p_{y}}-p_{y}\frac{\partial}{\partial p_{z}})(\frac{1}{\sqrt{2\pi\hbar}})^{3}\iiint exp[\frac{i}{\hbar}(\vec{p}-\vec{p}')\cdot\vec{r}]d\tau
  =(p_{z}\frac{\partial}{\partial p_{y}}-p_{y}\frac{\partial}{\partial p_{z}})\delta(\vec{p}-\vec{p}')$

  \item \textbf{晶格}  (1)三斜(1;$a_1\neq a_2\neq a_3;\alpha\neq\beta\neq\gamma$);单斜(2;$a_1\neq a_2\neq a_3;\alpha=\gamma=\pi/2\neq\beta$);
  正交(4;$a_1\neq a_2\neq a_3;\alpha=\beta=\gamma=\pi/2$);四角(2,$a_1=a_2\neq a_3;\alpha=\beta=\gamma=\pi/2$);
  立方(3;$a_1=a_2=a_3;\alpha=\beta=\gamma=\pi/2$);三角(1,$a_1=a_2=a_3;\alpha=\beta=\gamma\neq\pi/2$);
  六角(1;$a_1=a_2\neq a_3;\alpha=\beta=\pi/2,\gamma=2\pi/3$)(2)sc(简单立方);bcc(体心立方);fcc(面心立方);hcp(六角密堆积)
  (3)常见结构:NaCl($Cl^{-}$面心\&角+$Na^{+}$边中\&体心);CsCl($Cs^{+}$体心+$Cl^{-}$角);
  金刚石结构(fcc+$000\&\frac{1}{4}\frac{1}{4}\frac{1}{4}$);ZnS结构(Zn$000,0\frac{1}{2}\frac{1}{2},\frac{1}{2}0\frac{1}{2},\frac{1}{2}\frac{1}{2}0$;
  S$\frac{1}{4}\frac{1}{4}\frac{1}{4},\frac{1}{4}\frac{3}{4}\frac{3}{4},\frac{3}{4}\frac{1}{4}\frac{3}{4},,\frac{3}{4}\frac{3}{4}\frac{1}{4}$)

  \item \textbf{两种指标}设晶面截距为$a_1,a_2,a_3$(1)$(a_1^{-1},a_{2}^{-1},a_2^{-1})$;(2)$[a_1,a_2,a_3]$.上划线表示负号$[u\overline{v}w]$
  \item \textbf{布拉格条件}$2d\sin{\theta}=n\lambda;\Delta\vec{k}=\vec{G};2\vec{k}\cdot\vec{G}=\vec{G}^2$;
  \item \textbf{劳厄条件}$\vec{a_1}\cdot\Delta\vec{k}=2\pi v_1;\vec{a_2}\cdot\Delta\vec{k}=2\pi v_2;\vec{a_3}\cdot\Delta\vec{k}=2\pi v_3$;
  \item \textbf{倒格子初基平移矢量}$\vec{b_1}=2\pi\frac{\vec{a_2}\times\vec{a_3}}{\vec{a_1}\cdot\vec{a_2}\times\vec{a_3}},
  \vec{b_2}=2\pi\frac{\vec{a_3}\times\vec{a_1}}{\vec{a_1}\cdot\vec{a_2}\times\vec{a_3}},
  \vec{b_3}=2\pi\frac{\vec{a_1}\times\vec{a_2}}{\vec{a_1}\cdot\vec{a_2}\times\vec{a_3}}$

  \item \textbf{倒格矢}$\vec{G}=v_1\vec{b_1}+v_2\vec{b_2}+v_3\vec{b_3},v_{i}为\mathcal{Z}$
  \item \textbf{几何结构因子}前提:方向为$\vec{k'}=\vec{k}+\Delta\vec{k}=\vec{k}+\vec{G}$,
  $S_{G}=\sum_{j} f_{j}e^{-i\vec{r_{j}}\cdot\vec{G}}=\sum_{j}e^{-i2\pi(x_j v_1+y_j y_2+z_j v_3)}$,
  其中$f_{j}=\int dVn_{j}(\vec{r})e^{-i\vec{G}\cdot\vec{r}}$
  \item \textbf{第一布里渊区}倒格子的维格纳-塞茨原胞(1)sc$-\rangle$sc($2\pi/a$);bcc$-\rangle$棱形十二面体($2\pi/a\sqrt{2}$);
  fcc$-\rangle$截角八面体(八面体的每个角都被切下,使得相邻三个面的正方形的边能围成正六边形)

  \item \textbf{声子}
  (1)单原子:$u_{s\pm 1}=ue^{isKa}exp^{\pm iKa}$色散关系:$w^2=(2C/M)(1-\cos{Ka})$;
  $w^2=(4C/M)\sin^2{\frac{1}{2}Ka}$;群速:$v_{g}=\frac{dw}{dK}=\sqrt{\frac{Ca^2}{M}}\cos{\frac{1}{2}Ka}$;
  长波极限($Ka<<1$):$w^2=(C/M)K^{2}a^{2}$
  (2)双原子:原胞p个原子,3个声学支,3p-3个光学支.$M_1\frac{d^2 u_s}{dt^2}=C(v_s+v_{s-1}-2u_s);M_2\frac{d^2 v_s}{dt^2}=C(u_{s+1}+u_{s}-2v_s)$.
  $u_s=ue^{isKa}e^{-iwt},v_s=ve^{isKa}e^{-iwt}$,行列式系数为0:$M_1 M_2 w^4-2C(M_1+M_2)w^2+2C^2(1-\cos{Ka})=0$;长波极限:
  光学支$w^2=2C(\frac{1}{M_1}+\frac{1}{M_2})$,声学支$w^2=\frac{C}{2(M_1+M_2)}K^2 a^2$
\end{itemize}

% \begin{equation}
% \begin{aligned}
%   &A^2\int_{0}^{2\pi}\sin{x} ^4 dx = 1\\
%   &\Longrightarrow A=\sqrt{\frac{4}{3\pi}},\\
%   &\Psi(\phi,t=0)=A\sin ^2\phi = \frac{A}{2}-\frac{A}{2}\cos{2\phi},\\
%   &\Psi(x,t=0) = \sum_{i}c_{i}\psi_{i}(x,t=0),\\
%   &\Psi(x,t) = \sum_{i}c_{i}\psi_{i}(x,t=0)e^{-\frac{iE_{i}t}{\hbar}}\\\
%   &\longrightarrow \Psi(x,t) = \frac{A}{2}e^{-\frac{iE_{0}t}{\hbar}}-
%   \frac{A}{2}\cos(2\phi)e^{-\frac{iE_{2}t}{\hbar}}\\
% \end{aligned}
% \end{equation}

  % \begin{table}[htbp]
  %   \caption{布拉维格子及其倒格子的几何参数}
  %   \centering
  %   \begin{tabular}{c|c|c|c|c}
  %   \hline
  %   & 点群 & 简写 & 布拉维格子几何参数 & 倒格子几何参数 \\
  %   \hline
  %   \hline
  %   立方晶系 & m3m & fcc & a & $\frac{2\pi}{a}$ \\
  %   & & bcc & a & $\frac{2\pi}{a}\sqrt{\frac{2}{3}}$ \\
  %   \hline
  %   正交晶系 & mmm & & a, b, c & $\frac{2\pi}{a}$, $\frac{2\pi}{b}$, $\frac{2\pi}{c}$ \\
  %   \hline
  %   单斜晶系 & 2/m & & a, b, c, $\beta$ & $\frac{2\pi}{a}$, $\frac{2\pi}{b}$, $\frac{2\pi}{c}$, $\alpha^{},\beta^{},\gamma^{}$ \\
  %   \hline
  %   三斜晶系 & 1 & & a, b, c, $\alpha$, $\beta$, $\gamma$ & $\alpha^{},\beta^{} ,\gamma^{}$ \\
  %   \hline
  %   菱面体晶系 & m3m & & a, c & $\frac{2\pi}{a}$, $\frac{2\pi}{c}$ \\
  %   \hline
  %   正四面体晶系 & 23 & & a & $\frac{4\pi}{a}$ \\
  %   \hline
  %   正十二面体晶系 & m3 & & a & $\frac{4\pi}{a}$ \\
  %   \hline
  %   \end{tabular}
  %   \end{table}



\end{document}